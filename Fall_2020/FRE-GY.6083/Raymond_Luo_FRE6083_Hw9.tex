\documentclass[12pt,twoside, letter]{exam}
\usepackage{enumitem, kantlipsum}
\usepackage[margin=1in,left=1in,right=1in,top=1in,bottom=1in]{geometry}
\usepackage{graphicx,epstopdf}
\usepackage{amssymb,amsmath,amsfonts, amsthm}
\usepackage{wasysym}
\newtheorem{theorem}{Theorem}
\newtheorem{corollary}{Corollary}[theorem]
\newtheorem{lemma}[theorem]{Lemma}
\usepackage{hyperref}
\usepackage{tikz}
\usetikzlibrary{matrix}
\usepackage{xstring}
\usetikzlibrary{calc}
\usepackage{ducksay}
\newtheorem{prop}{Proposition}

\theoremstyle{definition}
\newtheorem{definition}{Definition}

\usepackage{bbm}
\usepackage{verbatim}
\usepackage{bbold}
\usepackage{phfparen}
\usepackage{sets}
\newcommand{\nn}{\mathbb{N}}
\newcommand{\rr}{\mathbb{R}}
\newcommand{\cc}{\mathbb{C}}
\newcommand{\cb}{\mathcal{B}}
\newcommand{\ctau}{\mathcal{T}}
\newcommand{\co}{\mathcal{O}}
\newcommand{\zz}{\mathbb{Z}}
\newcommand{\ee}{\mathbb{E}}
\newcommand{\qq}{\mathbb{Q}}
\newcommand{\interior}{\text{Int}}
\newcommand{\pp}{\mathbb{P}}
\newcommand{\id}{\mathbbm{1}}
\newcommand{\Co}{\text{Co}}
\newcommand{\Cl}{\text{Cl}}
\usepackage{mathtools}
\DeclarePairedDelimiter\ceil{\lceil}{\rceil}
\DeclarePairedDelimiter\floor{\lfloor}{\rfloor}


\usepackage{indentfirst}
\setlist{
    listparindent = \parindent,
    parsep = 6pt,
}

\makeatletter
\newsavebox\myboxA
\newsavebox\myboxB
\newlength\mylenA

\newcommand*\xoverline[2][0.9]{%
    \sbox{\myboxA}{$\m@th#2$}%
    \setbox\myboxB\null% Phantom box
    \ht\myboxB=\ht\myboxA%
    \dp\myboxB=\dp\myboxA%
    \wd\myboxB=#1\wd\myboxA% Scale phantom
    \sbox\myboxB{$\m@th\overline{\copy\myboxB}$}%  Overlined phantom
    \setlength\mylenA{\the\wd\myboxA}%   calc width diff
    \addtolength\mylenA{-\the\wd\myboxB}%
    \ifdim\wd\myboxB<\wd\myboxA%
       \rlap{\hskip 0.5\mylenA\usebox\myboxB}{\usebox\myboxA}%
    \else
        \hskip -0.5\mylenA\rlap{\usebox\myboxA}{\hskip 0.5\mylenA\usebox\myboxB}%
    \fi}
\makeatother

\usepackage{float}
\floatstyle{boxed}
\restylefloat{figure}

\printanswers

\begin{document}

\abovedisplayskip=12pt
\belowdisplayskip=12pt
\abovedisplayshortskip=7pt
\belowdisplayshortskip=10pt
\allowdisplaybreaks

\setlength{\parindent}{18pt}

\title{Quantitative Methods: Assignment 9}
\author{Raymond Luo}
\date{\today}
\maketitle

\begin{enumerate}
  \item Implement the Finite Difference method seen in class for computing the price
    of a European call. Use the parameters $T = 1, r= 0.01, \sigma = 0.2, K=100$.
    \begin{solution}
      \begin{figure}[H]
        \centering
          \includegraphics[width=6in]{Hw9}
      \end{figure}
    \end{solution}
  \item What value of $R$ did you choose? Why? Discuss.
    \begin{solution}
      We note that for our variables, we defined $y = log(x)$. Noting that $\sigma = 0.2, K = 100$,
      it is expected that we want to at least observe the behavior of prices around $[0,120]$ at maturity.
      I supposed that the a fair window of price observation was around $S(T) = \$150$ and noted that
      $\ln(150) \approx 5$ so I chose $R = 5$.
    \end{solution}
  \item Plot the value of the European call at time $0$ for a range of values of the underlying
    stock.
      \begin{solution}
        With values of $\delta_t = 0.1, \delta_y = 0.5$, and $R = 5$ we get: \\
        \begin{figure}[H]
          \centering
            \includegraphics[width=5in]{Hw9_b}
        \end{figure}
      \end{solution}
  \item Compare the computed value with the value obtained by using the analytical formula
    and compute the error. Refine the mesh 3 times by multiplying the number of spatial points
    by 2 and the number of time steps by 4, each time, and compute the corresponding error.
    Do you achieve a convergence of order 2 in space and of order one in time?
      \begin{solution}
        We compare the error of the previous solution of $\delta_t = 0.1, \delta_y = 0.5$, and $R = 5$ with the
        new errors by reducing $\delta_t$ by a ratio of $1/4$ and $\delta_y$ by a ratio of 1/2 simultaneously
        three times to get the following error graph: 
        \begin{figure}[H]
          \centering
            \includegraphics[width=5in]{Hw9_c}
        \end{figure}
        We can observe that our error approximately halves every time we refine our mesh. This means that we visually demonstrate
        that we achieve a convergence of order 2 in space and order 1 in time as we had refined our mesh by doubling the
        spatial factor and quadrupling the time factor.
      \end{solution}
\end{enumerate}


\end{document}
