\documentclass[12pt,twoside, letter]{exam}
\usepackage{enumitem, kantlipsum}
\usepackage[margin=1in,left=1in,right=1in,top=1in,bottom=1in]{geometry}
\usepackage{graphicx,epstopdf}
\usepackage{amssymb,amsmath,amsfonts, amsthm}
\usepackage{wasysym}
\newtheorem{theorem}{Theorem}
\newtheorem{corollary}{Corollary}[theorem]
\newtheorem{lemma}[theorem]{Lemma}
\usepackage{hyperref}
\usepackage{tikz}
\usepackage{xstring}
\usetikzlibrary{calc}
\usepackage{ducksay}
\newtheorem{prop}{Proposition}

\theoremstyle{definition}
\newtheorem{definition}{Definition}

\usepackage{bbm}
\usepackage{verbatim}
\usepackage{bbold}
\usepackage{phfparen}
\usepackage{sets}
\newcommand{\nn}{\mathbb{N}}
\newcommand{\rr}{\mathbb{R}}
\newcommand{\cc}{\mathbb{C}}
\newcommand{\cb}{\mathcal{B}}
\newcommand{\ctau}{\mathcal{T}}
\newcommand{\co}{\mathcal{O}}
\newcommand{\zz}{\mathbb{Z}}
\newcommand{\ee}{\mathbb{E}}
\newcommand{\qq}{\mathbb{Q}}
\newcommand{\interior}{\text{Int}}
\newcommand{\pp}{\mathbb{P}}
\newcommand{\id}{\mathbbm{1}}
\newcommand{\Co}{\text{Co}}
\newcommand{\Cl}{\text{Cl}}
\usepackage{mathtools}
\DeclarePairedDelimiter\ceil{\lceil}{\rceil}
\DeclarePairedDelimiter\floor{\lfloor}{\rfloor}


\usepackage{indentfirst}
\setlist{
    listparindent = \parindent,
    parsep = 6pt,
}

\makeatletter
\newsavebox\myboxA
\newsavebox\myboxB
\newlength\mylenA

\newcommand*\xoverline[2][0.9]{%
    \sbox{\myboxA}{$\m@th#2$}%
    \setbox\myboxB\null% Phantom box
    \ht\myboxB=\ht\myboxA%
    \dp\myboxB=\dp\myboxA%
    \wd\myboxB=#1\wd\myboxA% Scale phantom
    \sbox\myboxB{$\m@th\overline{\copy\myboxB}$}%  Overlined phantom
    \setlength\mylenA{\the\wd\myboxA}%   calc width diff
    \addtolength\mylenA{-\the\wd\myboxB}%
    \ifdim\wd\myboxB<\wd\myboxA%
       \rlap{\hskip 0.5\mylenA\usebox\myboxB}{\usebox\myboxA}%
    \else
        \hskip -0.5\mylenA\rlap{\usebox\myboxA}{\hskip 0.5\mylenA\usebox\myboxB}%
    \fi}
\makeatother

\usepackage{float}
\floatstyle{boxed}
\restylefloat{figure}

\printanswers

\begin{document}


\abovedisplayskip=12pt
\belowdisplayskip=12pt
\abovedisplayshortskip=7pt
\belowdisplayshortskip=10pt
\allowdisplaybreaks

\setlength{\parindent}{18pt}

\title{Quantitative Methods: Assignment 5}
\author{Raymond Luo}
\date{\today}
\maketitle

\noindent {\bf Problem 1 (30 points):}
  \par{We consider the 3-period name model seen in class with the parameters}
\begin{equation*}
  S_0 = 4, u = 2, d = 1/2, r = 1/5, K = 5
\end{equation*}
The price fo the underlying asset is denoted by $S_n$ for $n = 0,1,2,3$

\begin{enumerate}
  \item Compute the price of $V_0$, at time 0, of a European call option on the underlying asset $S_n$ with payoff
    $V_3 = (S_3 - 2)^+$ at time 3.
    \begin{solution}
      We first note that the arbitrage free pricing of the option should be the expected 
    \end{solution}
  \item Give also the hedging strategy $\Delta_0$ at time 0.
  \item Compute now the price of a Lookback option with payoff
    \begin{equation*}
      V_3  =\max_{0 \leq n \leq 3}(S_n - S_3)
    \end{equation*}
    for the same set of parameters
  \item Give the hedging strategy $\Delta_0$ for the lookback option.
\end{enumerate}

\noindent {\bf Problem 2 (10 points):}
  \par{Consider the one-period binomial tree model seen in class with $d = 1/2, u = 1.5, r = 1, S_0 = 3$. Is there an arbitrage
    opportunity in this model? Can you exhibit such a strategy?}

\noindent {\bf Problem 3 (10 points):}
  \par{Consider the general one-period binomial tree model seen in class with payoff $V_1 = S_1$ (in other words we set the strike price
  to 0, K = 0)}
    \begin{enumerate}
      \item Write the risk-neutral pricing formula giving the price $V_0$ at time 0, of this claim
      \item What is the price?
    \end{enumerate}

\noindent {\bf Problem 4 (50 points) Asian Option:}
  \par{Consider the 3-period model seen in class with $d = 1/2, u = 2, r = 1/4, S_0 = 4$ and consider the Asian option with payoff}
    \begin{equation*}
      V_3 = (\frac{1}{4}Y_3 - 4)^+
    \end{equation*}
  where the variable $Y_n$ is defined as
    \begin{equation*}
      Y_n = \sum^n_{k=0} S_k
    \end{equation*}
    \begin{enumerate}
      \item Write a recursive formula for $V_n$ in terms of $V_{n+1}$ that will allow you to compute the price of the Asian option.
      \item Provide a formula for the number of shares $\Delta_n = \delta_n(s,y)$ of underlying asset that should be held in the replicating portfolio at
        time $n$ for $n = 0,1,2$
      \item Use the algorithm developed in the first question to compute the price of the Asian option $V_0$ at time 0.
    \end{enumerate}
\noindent {\bf Optional Problem (0 credit):} Exercise 1.9 page 22 in the textbook by Shreve.

\end{document}
