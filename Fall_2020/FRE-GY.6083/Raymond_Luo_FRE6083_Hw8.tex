\documentclass[12pt,twoside, letter]{exam}
\usepackage{enumitem, kantlipsum}
\usepackage[margin=1in,left=1in,right=1in,top=1in,bottom=1in]{geometry}
\usepackage{graphicx,epstopdf}
\usepackage{amssymb,amsmath,amsfonts, amsthm}
\usepackage{wasysym}
\newtheorem{theorem}{Theorem}
\newtheorem{corollary}{Corollary}[theorem]
\newtheorem{lemma}[theorem]{Lemma}
\usepackage{hyperref}
\usepackage{tikz}
\usetikzlibrary{matrix}
\usepackage{xstring}
\usetikzlibrary{calc}
\usepackage{ducksay}
\newtheorem{prop}{Proposition}

\theoremstyle{definition}
\newtheorem{definition}{Definition}

\usepackage{bbm}
\usepackage{verbatim}
\usepackage{bbold}
\usepackage{phfparen}
\usepackage{sets}
\newcommand{\nn}{\mathbb{N}}
\newcommand{\rr}{\mathbb{R}}
\newcommand{\cc}{\mathbb{C}}
\newcommand{\cb}{\mathcal{B}}
\newcommand{\ctau}{\mathcal{T}}
\newcommand{\co}{\mathcal{O}}
\newcommand{\zz}{\mathbb{Z}}
\newcommand{\ee}{\mathbb{E}}
\newcommand{\qq}{\mathbb{Q}}
\newcommand{\interior}{\text{Int}}
\newcommand{\pp}{\mathbb{P}}
\newcommand{\id}{\mathbbm{1}}
\newcommand{\Co}{\text{Co}}
\newcommand{\Cl}{\text{Cl}}
\usepackage{mathtools}
\DeclarePairedDelimiter\ceil{\lceil}{\rceil}
\DeclarePairedDelimiter\floor{\lfloor}{\rfloor}


\usepackage{indentfirst}
\setlist{
    listparindent = \parindent,
    parsep = 6pt,
}

\makeatletter
\newsavebox\myboxA
\newsavebox\myboxB
\newlength\mylenA

\newcommand*\xoverline[2][0.9]{%
    \sbox{\myboxA}{$\m@th#2$}%
    \setbox\myboxB\null% Phantom box
    \ht\myboxB=\ht\myboxA%
    \dp\myboxB=\dp\myboxA%
    \wd\myboxB=#1\wd\myboxA% Scale phantom
    \sbox\myboxB{$\m@th\overline{\copy\myboxB}$}%  Overlined phantom
    \setlength\mylenA{\the\wd\myboxA}%   calc width diff
    \addtolength\mylenA{-\the\wd\myboxB}%
    \ifdim\wd\myboxB<\wd\myboxA%
       \rlap{\hskip 0.5\mylenA\usebox\myboxB}{\usebox\myboxA}%
    \else
        \hskip -0.5\mylenA\rlap{\usebox\myboxA}{\hskip 0.5\mylenA\usebox\myboxB}%
    \fi}
\makeatother

\usepackage{float}
\floatstyle{boxed}
\restylefloat{figure}

\printanswers

\begin{document}

\abovedisplayskip=12pt
\belowdisplayskip=12pt
\abovedisplayshortskip=7pt
\belowdisplayshortskip=10pt
\allowdisplaybreaks

\setlength{\parindent}{18pt}

\title{Quantitative Methods: Assignment 8}
\author{Raymond Luo}
\date{\today}
\maketitle

\noindent{\bf Problem 1}
  \begin{enumerate}
    \item Implement in Matlab the Monte Carlo Algorithm seen in class for pricing a European call option
      \begin{solution}
        \begin{figure}[H]
          \centering
            \includegraphics[width=3in]{Hw8_1a}
        \end{figure}
      \end{solution}
    \item Use your algorithm to compute the value at time 0 of a European call with maturity $T = 1$ year and
      strike price $K = 100$, for the current stock price $S_0 = 110$, assuming that the annualized parameters are
      $r = 0.01$, $\sigma = 0.2$.
        \begin{solution}
          Inputting $S_0 = 110, K = 100, T = 1, r = 0.01, \sigma = 0.02$, we get the following answers from the algorithm above: \\
          $n = 100,000$: $\hat{C} = 14.9452$ \\
          $n = 1,000,000$: $\hat{C} = 14.9540$
        \end{solution}
    \item Compare your above result with the exact value provided by the analytical formula seen in class for
      $n = 100,000$ and $n = 1,000,000$ sample paths.
        \begin{solution}
          We use the analytical closed form solution of the BS formula to find that for:\\
          \begin{equation*}
            S_0 = 110, K = 100, T = 1, r = 0.01, \sigma = 0.2
          \end{equation*}
          We have that $d_1 = \frac{\ln{S/K} + (r + \sigma^2/2)(T-t)}{\sigma\sqrt{T-t}} = \frac{\ln{1.1} + (0.01 + (0.2)^2/2)}{0.2} = 0.62655$,
          $d_2 = d_1 - \sigma\sqrt{T-t} = d_1 - 0.2 = 0.42655$ \\
          $C(S_1, t) = S_0 \Phi(d_1) - Ke^{-r(T-t)} \Phi(d_2) = 110\cdot 0.7345 - 100\cdot e^{-0.01} \cdot 0.6651 = 14.947$
        \end{solution}
  \par{A cash-or-nothing Binary pays nothing if the price ends up below the strike price and pays fixed amount $Q$ if it ends
    up above the strike price. Its payoff function is thus}
    \begin{equation*}
      V_T = \chi_{\{S_T \geq K \}} Q
    \end{equation*}
  \par{where $\chi_{\{S_T \geq K \}}$, the indicator function, is defined by }
    \begin{align*}
      \chi_{\{S_T \geq K \}} =
      \begin{cases}
        1 \quad \text{ if $S_T \geq K$} \\
        0 \quad \text{ Otherwise}
      \end{cases}
    \end{align*}

    \item Modify your algorithm to compute the price of a Binary call option with a cash-or-nothing payoff.
        \begin{solution}
          \begin{figure}[H]
            \centering
              \includegraphics[width=3in]{Hw8_1b}
          \end{figure}
        \end{solution}
    \item Compute the value of the call at time 0 for the following parameters:
      \begin{equation*}
        K = 100, T = 1, \sigma = 0.20, Q = 1, S_0 = 110, r = 0.01
      \end{equation*}

      \begin{solution}
        Running the above algorithm with $K = 100, T = 1, \sigma = 0.20, Q = 1, S_0 = 110, r = 0.01$ and
        $n = 1,000,000$, we get $\chi_{\{S_T \geq K\}} = 0.6582$
      \end{solution}
    \item Derive a price in closed form for the cash-or-nothing call at time $t$ when the value of the stock is $s$.
      \begin{solution}\\
        *We showed in class that the discounted option value is a martingale by extending the martingale nature of the discounted stock price.  \\
        We have that $v(s,t) = \ee^{\mathbb{Q}}[e^{-r(T-t)}V(T) | S(t) = s]$ and that: \\
        $S(T) = S(t)exp\{\sigma(\tilde{W}(T) - W(t)) + (r- \frac{1}{2}\sigma^2)(T-t)\}
        = S(t)exp\{\sigma(\sqrt{T-t}Y) + (r- \frac{1}{2}\sigma^2)(T-t)\}$ \\
        for $Y = -\frac{\tilde{W}(T)-W(t)}{\sqrt{T-t}}$.
        It then follows that:
        \begin{align*}
          v(s,t) = \ee^{\mathbb{Q}}[e^{-r(T-t)}\chi_{\{S_T \geq K \}} | S(t) = s] \\
          = e^{-r(T-t)}\frac{1}{\sqrt{2\pi}}\bigg(\int_{\rr} \id_{S(T) > K} e^{-y^2/2} dy \bigg)
        \end{align*}
        Note that $S(T) \geq K \Rightarrow \sigma(\sqrt{T-t}Y) + (r- \frac{1}{2}\sigma^2)(T-t) \geq \ln(\frac{K}{S(0)}) \\
        \Rightarrow Y \geq d_2 = \frac{\ln(\frac{S(0)}{K}) + (r- \frac{1}{2}\sigma^2)(T-t)}{\sigma(\sqrt{T-t})}$. \\
        We can then write the indicator function into the bounds of our integral to get:
        \begin{align*}
          v(s,t) = e^{-r(T-t)}\frac{1}{\sqrt{2\pi}}\bigg(\int^{\infty}_{d_2} e^{-y^2/2} dy \bigg) \\
          = e^{-r(T-t)}\Phi(d_2)
        \end{align*}
      \end{solution}
    \item Compute the numerical value of this call option for the parameters
      \begin{equation*}
        r = 0.01, S_0 = 110, K = 100, T = 1, \sigma = 0.2, Q = 1;
      \end{equation*}
      by using the analytical formula you derived in the previous question
        \begin{solution}
          We have that:\\
          $d_2 = \frac{\ln{S/K} + (r - \sigma^2/2)(T-t)}{\sigma\sqrt{T-t}} = \frac{\ln{1.1} + (0.01 + (0.2)^2/2)}{0.2} = 0.62655$ \\
          $v(s,t) = e^{-r(T-t)}\Phi(d_2) = 0.6585$
        \end{solution}
    \item Compare your answer with the answer you obtained by using Monte Carlo simulations.
      \begin{solution}
        The answers differ on the fourth digit, meaning error is of the fourth order. This agrees with the fact that the Monte Carlo Algorithm has error
        $O(\sqrt{n})$.
      \end{solution}
  \end{enumerate}

  \noindent{\bf Problem 2}
    \par{Consider a stock whose price per share is modeled by using the standard Geometric Brownian motion model}
  \begin{equation*}
    dS(t) = rS(t)dt + \sigma S(t)d\tilde{W}(t)
  \end{equation*}
  where $\tilde{W}(t)$ is a standard Brownian motion under the risk neutral-probability measure, $r > 0$ is the interest rate
  and $\sigma > 0$ is the volatility. Consider now a contingent claim on the underlying stock with maturity date $T$ and payoff

  \begin{equation*}
    V(T) = aS(T) + b
  \end{equation*}
  where $a,b$ are two positive real numbers.

  \par{Compute in closed-form the arbitrage free price, $v(t,s)$ of this claim at time $t$, for the current value of the stock $s$}

  \begin{solution}\\
    We showed in class that the discounted option value is a martingale by extending the martingale nature of the discounted stock price. \\
    We note that in an arbitrage-free market, discounted stock prices are a martingale under the standard probability measure. It then follows
    that as according to the fundamental theorem of asset pricing, we have an equivalent measure $\mathbb{Q}$ for our contingent claim, we have that:
    \begin{align*}
      v(s,t) &= \ee^{\mathbb{Q}}[e^{-r(T-t)}V(T) | S(t) = s] \\
      &= \ee^{\mathbb{Q}}[e^{-r(T-t)}\bigg(aS(T) + b \bigg) | S(t) = s]
      &= a \ee^{\mathbb{Q}}[e^{-r(T-t)}S(T)| S(t) = s] + be^{-r(T-t)} \\
      &= as + be^{-r(T-t)}
    \end{align*}
  \end{solution}

  \noindent{\bf Problem 3}
    \par{The price of a share of a stock, $S(t)$, satisfies the Stochastic Differential Equation}

    \begin{equation*}
      dS(t) = rS(t)dt + \sigma S(t)dW(t)
    \end{equation*}
    where $W$ is a standard Brownian motion, $r > 0$ is the risk-free rate (short rate) and $\sigma > 0$ is the volatility coefficient.
    \par{ Next we consider the Bull call spread option strategy that pays at maturity date the payoff}

    \begin{align*}
      V(T) =
      \begin{cases}
        B & \text{if $S(T) > B$} \\
        \frac{A+B}{B-A}S(T) - \frac{2AB}{B-A} & \text{ if $A \leq S(T) \leq B$} \\
        -A & \text{ if $S(T) < A$}
      \end{cases}
    \end{align*}
    where $0 < A < B$ are constant.

    \begin{enumerate}
      \item Write the risk-neutral pricing formula for the price $v(t,s)$ at time $t$, for the current stock price $S(t) = s$ of the Bull call spread.
        \begin{solution}
          \\
          We note that in an arbitrage-free market, discounted stock prices are a martingale under the standard probability measure. It then follows
          that as according to the fundamental theorem of asset pricing, we have an equivalent measure $\mathbb{Q}$ for our contingent claim, we have that:
          \begin{align*}
            v(s,t) = \ee^{\mathbb{Q}}[e^{-r(T-t)}V(T) | S(t) = s] \\
          \end{align*}
            The bull call spread can be constructed by buying a call option (with price $c_1(0,S_0)$) with a lower strike price $K_1$
            and selling another call option (with price $c_2(0,S_0)$) with a higher strike price $K_2$. We note that in this construction, the payoff is different
            as our payoff function seems to implicitly include a premia for buying and selling the options and then rescaling the payoffs. We can view our
            bull call spread as a linearly transformed version of the standard bull call spread by scaling of $\frac{B+A}{B-A}$ and paying initial premia $A$.
            We can verify this as the strike prices of our two component call options should be at the transition points for our function and that
            the lower bound of the payoff is the premia received by selling an option minus the premia used to buy an option. We have that:
            \begin{align*}
              \begin{cases}
                [c_2(0,S_0) - c_1(0,S_0)]e^{rT} = -A \\
                K_1 = A \\
                K_2 = B
              \end{cases}
            \end{align*}
            In this case, we would have payoff:
            \begin{align*}
              V(T) =
              \begin{cases}
                C\big[(S-A) - (S-B)\big] + D = C(B-A)+D & \text{if $S(T) > B$} \\
                C\big[0\big] + D = -A & \text{ if $S(T) < A$}
              \end{cases}
            \end{align*}
            From which we can get $C = \frac{B+A}{B-A}$ and $D = -A$. \\
            It follows that for $C_1, C_2$ as the value of our two component options, we have:
            \begin{align*}
              v(s,t) = \ee^{\mathbb{Q}}[e^{-r(T-t)}\bigg(\frac{B+A}{B-A}\big(C_1(T) + C_2(T)\big)-A\bigg) | S(t) = s] \\
              = \frac{B+A}{B-A}\ee^{\mathbb{Q}}[e^{-r(T-t)}C_2(T) | S(t) = s] - \frac{B+A}{B-A}\ee^{\mathbb{Q}}[e^{-r(T-t)}C_2(T) | S(t) = s] - Ae^{-r(T-t)}\\
              = \frac{B+A}{B-A}(c_2(s,t) - c_1(s,t)) - Ae^{-r(T-t)}
            \end{align*}
            Where $c_2$ is the value of the second call option sold with strike price $B$ and $c_1$ is the value of the first call option
            bought with strike price $A$. We can find analytic forms of both equations using Black-Scholes.
        \end{solution}
      \item Derive an analytical formula for the price $v(t,s)$ of the Bull call spread. Show your work!
        \begin{solution}
          We can proceed through the risk neutral approach by noticing that our payoff [which represents our terminal condition] $V(T) = (S - A)^+ - (S - B)^+ -A$
          [that is the difference of the payoff of the two calls and including cost of premia for the two options].\\
          We also have that $S(T) = S(t)exp\{\sigma(\tilde{W}(T) - W(t)) + (r- \frac{1}{2}\sigma^2)(T-t)\}
          = S(t)exp\{\sigma(\sqrt{T-t}Y) + (r- \frac{1}{2}\sigma^2)(T-t)\}$ \\
          for $Y = -\frac{\tilde{W}(T)-W(t)}{\sqrt{T-t}}$.
          We can rewrite this terminal condition as $V(T) = B \cdot \id_{S(T) > B} + \bigg(\frac{A+B}{B-A}S(T) - \frac{2AB}{B-A}\bigg)\cdot \id_{A \leq S(T) \leq B}
          - A \cdot \id_{S(T) < A} - A$. We proceed with a risk-neutral measure approach with:
          \begin{align*}
            v(s,t) &= \ee^{\mathbb{Q}}[e^{-r(T-t)}\chi_{\{S_T \geq K \}} | S(t) = s] \\
            &= e^{-r(T-t)}\frac{1}{\sqrt{2\pi}}\bigg(\int_{\rr} B \cdot \id_{S(T) > B} + \\
            &+ \bigg(\frac{A+B}{B-A}S(T) - \frac{2AB}{B-A}\bigg)\cdot \id_{A \leq S(T) \leq B}
            - A \cdot \id_{S(T) < A} - A e^{-y^2/2} dy \bigg)
          \end{align*}
          We note that $S(T) \geq B \Rightarrow \sigma(\sqrt{T-t}Y) + (r- \frac{1}{2}\sigma^2)(T-t) \geq \ln(\frac{K}{S(0)}) \\
          \Rightarrow Y \geq d_2 = \frac{\ln(\frac{S(0)}{B}) + (r- \frac{1}{2}\sigma^2)(T-t)}{\sigma(\sqrt{T-t})}$ \\
          Likewise, $S(T) < A \Rightarrow Y < d_1 = \frac{\ln(\frac{S(0)}{A}) + (r- \frac{1}{2}\sigma^2)(T-t)}{\sigma(\sqrt{T-t})}$ so that
          we have integral form:
          \begin{align*}
            v(s,t) &= e^{-r(T-t)}\frac{1}{\sqrt{2\pi}}\bigg(\int^{\infty}_{d_2} B e^{-y^2/2} dy \bigg) + \\
            &\qquad + e^{-r(T-t)}\frac{1}{\sqrt{2\pi}}\bigg(\int^{d_2}_{d_1} \bigg(\frac{A+B}{B-A}S(T) - \frac{2AB}{B-A}\bigg) e^{-y^2/2} dy \bigg)\\
            &\qquad + e^{-r(T-t)}\frac{1}{\sqrt{2\pi}}\bigg(\int^{d_1}_{-\infty} -A e^{-y^2/2} dy \bigg) \\
            &= e^{-r(T-t)}\frac{1}{\sqrt{2\pi}}\bigg(\int^{\infty}_{d_2} B+A e^{-y^2/2} dy \bigg) + \\
            &\qquad+ e^{-r(T-t)}\frac{1}{\sqrt{2\pi}}\bigg(\int^{d_2}_{d_1} \bigg(\frac{A+B}{B-A}S(T) - A\frac{B+A}{B-A}\bigg) e^{-y^2/2} dy \bigg)\\
            &\qquad+ e^{-r(T-t)}\frac{1}{\sqrt{2\pi}}\bigg(\int^{\infty}_{-\infty} -A e^{-y^2/2} dy \bigg) \\
            &= \frac{B+A}{B-A} \bigg[e^{-r(T-t)}\frac{1}{\sqrt{2\pi}}\bigg(\int^{\infty}_{d_2} B-A e^{-y^2/2} dy \bigg) + \\
            &\qquad+ e^{-r(T-t)}\frac{1}{\sqrt{2\pi}}\bigg(\int^{d_2}_{d_1} \bigg(S(T) - A\bigg) e^{-y^2/2} dy \bigg)\bigg] - Ae^{-r(T-t)}\\
            &= \frac{B+A}{B-A} \bigg[e^{-r(T-t)}\frac{1}{\sqrt{2\pi}}\bigg(-\int^{\infty}_{d_2}  \bigg(S(T) - B\bigg)e^{-y^2/2} dy \bigg) + \\
            &\qquad+ e^{-r(T-t)}\frac{1}{\sqrt{2\pi}}\bigg(\int^{\infty}_{d_1} \bigg(S(T) - A\bigg) e^{-y^2/2} dy \bigg)\bigg] - Ae^{-r(T-t)}\\
          \end{align*}
          It's clear to see that the two integrals inside simplify into the Black-scholes formulas for call options $c_2, c_1$ respectively.
        \end{solution}
    \end{enumerate}


\end{document}
