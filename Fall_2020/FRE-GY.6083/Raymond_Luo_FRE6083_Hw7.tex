\documentclass[12pt,twoside, letter]{exam}
\usepackage{enumitem, kantlipsum}
\usepackage[margin=1in,left=1in,right=1in,top=1in,bottom=1in]{geometry}
\usepackage{graphicx,epstopdf}
\usepackage{amssymb,amsmath,amsfonts, amsthm}
\usepackage{wasysym}
\newtheorem{theorem}{Theorem}
\newtheorem{corollary}{Corollary}[theorem]
\newtheorem{lemma}[theorem]{Lemma}
\usepackage{hyperref}
\usepackage{tikz}
\usetikzlibrary{matrix}
\usepackage{xstring}
\usetikzlibrary{calc}
\usepackage{ducksay}
\newtheorem{prop}{Proposition}

\theoremstyle{definition}
\newtheorem{definition}{Definition}

\usepackage{bbm}
\usepackage{verbatim}
\usepackage{bbold}
\usepackage{phfparen}
\usepackage{sets}
\newcommand{\nn}{\mathbb{N}}
\newcommand{\rr}{\mathbb{R}}
\newcommand{\cc}{\mathbb{C}}
\newcommand{\cb}{\mathcal{B}}
\newcommand{\ctau}{\mathcal{T}}
\newcommand{\co}{\mathcal{O}}
\newcommand{\zz}{\mathbb{Z}}
\newcommand{\ee}{\mathbb{E}}
\newcommand{\qq}{\mathbb{Q}}
\newcommand{\interior}{\text{Int}}
\newcommand{\pp}{\mathbb{P}}
\newcommand{\id}{\mathbbm{1}}
\newcommand{\Co}{\text{Co}}
\newcommand{\Cl}{\text{Cl}}
\usepackage{mathtools}
\DeclarePairedDelimiter\ceil{\lceil}{\rceil}
\DeclarePairedDelimiter\floor{\lfloor}{\rfloor}


\usepackage{indentfirst}
\setlist{
    listparindent = \parindent,
    parsep = 6pt,
}

\makeatletter
\newsavebox\myboxA
\newsavebox\myboxB
\newlength\mylenA

\newcommand*\xoverline[2][0.9]{%
    \sbox{\myboxA}{$\m@th#2$}%
    \setbox\myboxB\null% Phantom box
    \ht\myboxB=\ht\myboxA%
    \dp\myboxB=\dp\myboxA%
    \wd\myboxB=#1\wd\myboxA% Scale phantom
    \sbox\myboxB{$\m@th\overline{\copy\myboxB}$}%  Overlined phantom
    \setlength\mylenA{\the\wd\myboxA}%   calc width diff
    \addtolength\mylenA{-\the\wd\myboxB}%
    \ifdim\wd\myboxB<\wd\myboxA%
       \rlap{\hskip 0.5\mylenA\usebox\myboxB}{\usebox\myboxA}%
    \else
        \hskip -0.5\mylenA\rlap{\usebox\myboxA}{\hskip 0.5\mylenA\usebox\myboxB}%
    \fi}
\makeatother

\usepackage{float}
\floatstyle{boxed}
\restylefloat{figure}

\printanswers

\begin{document}

\abovedisplayskip=12pt
\belowdisplayskip=12pt
\abovedisplayshortskip=7pt
\belowdisplayshortskip=10pt
\allowdisplaybreaks

\setlength{\parindent}{18pt}

\title{Quantitative Methods: Assignment 7}
\author{Raymond Luo}
\date{\today}
\maketitle


\noindent{ \bf Problem 1}
  \par{Compute the differentials of:}

  \begin{enumerate}
    \item $W^2(t)$
      \begin{solution}
        We do this with usage of the product rule in stochastic calculus:
        \begin{align*}
          d(W^2(t)) = d(W(t)\cdot W(t)) = dW(t)W(t) + W(t)dW(t) + dW(t)dW(t) \\
          = 2W(t) dW(t) + dt
        \end{align*}
      \end{solution}
    \item $W^3(t)$
      \begin{solution}
        We proceed with ito's lemma for $f(x,t) = x^3$ so that:
        \begin{align*}
          d(W^3(t)) = f'(W(t)) dW(t) + \frac{1}{2} f''(W(t)) (dW(t))^2 \\
          = 3W^2(t) dW(t)  + 3W(t) dt
        \end{align*}
      \end{solution}
    \item $\exp{\{-rt\}}W(t)$
      \begin{solution}
        We proceed with ito's lemma for $f(x,t) = xe^{-rt}$ so that:
        \begin{align*}
          d(e^{-rtW(t)}) = f_{t} dt + f_{x} dW(t) + \frac{1}{2} f_{xx} dt \\
          = -rW(t)e^{-rt} dt + e^{-rt} dW(t) + \frac{1}{2} 0 (dW(t))^2 \\
          = -rW(t)e^{-rt} dt + e^{-rt} dW(t)
        \end{align*}
      \end{solution}
    \item $\exp{\{t^2 - W(t)\}}$
      \begin{solution}
        We proceed with ito's lemma for $f(x,t) = e^{t^2 - x}$ so that:
        \begin{align*}
          d(e^{t^2 - W(t)}) = f_{t} dt + f_{x} dW(t) + \frac{1}{2} f_{xx} dt \\
          = \big(2t\cdot e^{t^2 - W(t)} + \frac{1}{2}e^{t^2 - W(t)} \big)dt
          - e^{t^2 - W(t)} dW(t) \\
          = (2t + \frac{1}{2})e^{t^2 - W(t)} dt - e^{t^2 - W(t)} dW(t)
        \end{align*}
      \end{solution}
  \end{enumerate}

\noindent{ \bf Problem 2}
  \begin{enumerate}
    \item Compute the differential of $W^4(t)$, where $W$ is a standard Brownian motion.
      \begin{solution}
        We proceed with ito's lemma for $f(x,t) = x^4$ so that:
        \begin{align*}
          d(W^4(t)) = f' dW(t) + \frac{1}{2} f'' dt  = 4W^3(t) dW(t) + 6W^2(t) dt
        \end{align*}
      \end{solution}
    \item Integrate the above formula on $[0,T]$.
      \begin{solution}
        \begin{align*}
          \int^T_0 d(W^4(t)) &= \int^T_0 4W^3(t) dW(t) + \int^T_0 6W^2(t) dt \\
          W^4(T) - W^4(0) &= \int^T_0 4W^3(t) dW(t) + \int^T_0 6W^2(t) dt \\
          W^4(T) &= \int^T_0 4W^3(t) dW(t) + \int^T_0 6W^2(t) dt
        \end{align*}
      \end{solution}
    \item Take the expectation of the left and right-hand sides and deduce $\ee[W^4(T)]$
      \begin{solution}
        On the left-hand side, we have that:
        \begin{align*}
          \ee[\int^T_{0} d(W^4(t))] = \ee[W^4(T)- W^4(0)] = \ee[W^4(T)]
        \end{align*}
        On the right-hand side, we have that:
        \begin{align*}
          &\ee[\int^T_0 4W^3(t) dW(t) + \int^T_0 6W^2(t) dt] \\
          &= \ee[\int^T_0 4W^3(t) dW(t)] + \ee[\int^T_0 6W^2(t) dt] \\
          &= 0 + \int^T_0 6\ee[W^2(t)] dt \text{ By Fubini's theorem in Classical Calculus} \\
          &= \int^T_0 6t dt = 3T^2
        \end{align*}
      \end{solution}
  \end{enumerate}

\noindent{ \bf Problem 3}
  \par{We consider the following mean-reverting model for the spread $S(t)$ of
  two co-integrated stocks:}
    \begin{equation*}
      dS(t) = -\lambda S(t) dt + \sigma dW(t), S(0) = s > 0
    \end{equation*}
where $\lambda > 0, \sigma > 0$

\begin{enumerate}
  \item Show by using Ito-Doeblin formula that
    \begin{equation*}
      S(t) = e^{-\lambda t}(s + \sigma \int^t_0 e^{\lambda u} dW(u))
    \end{equation*}
    is a solution of the above Stochastic Differential Equation.

    \begin{solution}
      \\
      We let $f(t,x) = e^{\lambda t}\cdot x$ and then use the Ito-Doeblin formula to find that:
      \begin{align*}
        d(f(t,S(t)) &= d(e^{\lambda t}\cdot S(t)) = \lambda e^{\lambda t} \cdot S(t) dt + e^{\lambda t} dS(t) + \frac{1}{2} \cdot 0 (dS(t))^2 \\
        &= \lambda e^{\lambda t} \cdot S(t) dt + e^{\lambda t} (-\lambda S(t) dt + \sigma dW(t)) \\
        &= e^{\lambda t} \sigma dW(t)
      \end{align*}
      If we integrate the above expression over interval [0,t], we get:
      \begin{align*}
        &e^{\lambda t}\cdot S(t) - e^{\lambda 0}\cdot S(0) = \int^t_0 e^{\lambda u} \sigma dW(u) \\
        &e^{\lambda t} S(t) = S(0) + \int^t_0 e^{\lambda u} \sigma dW(u) \\
        &S(t) = e^{-\lambda t}\big(s + \sigma \int^t_0 e^{\lambda u}  dW(u) \big), \text{where $S(0) = s > 0$ }
      \end{align*}
    \end{solution}
  \item Give the distribution of $S(t)$ and specify its mean and variance.
    \begin{solution}
      We have:
      \begin{align*}
        \ee[S(t)] = \ee[e^{-\lambda t}\big(s + \sigma \int^t_0 e^{\lambda u}  dW(u) \big)] \\
        = e^{-\lambda t}s + \ee[e^{-\lambda t}\sigma \int^t_0 e^{\lambda u}  dW(u)] \\
        = e^{-\lambda t}s + e^{-\lambda t}\sigma \ee[ \int^t_0 e^{\lambda u}  dW(u)] \\
        = e^{-\lambda t}s + e^{-\lambda t}\sigma\cdot 0 = e^{-\lambda t}s
      \end{align*}
      The variance follows from:
      \begin{align*}
        Var[S(t)] &= Var[e^{-\lambda t}\big(s + \sigma \int^t_0 e^{\lambda u}  dW(u) \big)] \\
        &= Var[e^{-\lambda t}\sigma \int^t_0 e^{\lambda u}  dW(u)] \\
        &= e^{-2\lambda t}\sigma^2 Var[\int^t_0 e^{\lambda u}  dW(u)] \\
        &= e^{-2\lambda t}\sigma^2 \bigg( \ee[(\int^t_0 e^{\lambda u}  dW(u))^2] - \ee[\int^t_0 e^{\lambda u}  dW(u)]^2 \bigg) \\
        &= e^{-2\lambda t}\sigma^2 \bigg( \ee[(\int^t_0 e^{2\lambda u}  du] - (0)^2 \bigg) \\
        &= e^{-2\lambda t}\sigma^2 \bigg( \frac{1}{2\lambda} e^{2\lambda t} - \frac{1}{2\lambda} \bigg) \\
        &= \sigma^2 \bigg( \frac{1}{2\lambda} - \frac{1}{2\lambda}e^{-2\lambda t} \bigg)
      \end{align*}

    \end{solution}
  \item Give $\lim_{t \rightarrow +\infty} \ee[S(t)]$ and interpret the result in plain english
    \begin{solution}
      \begin{align*}
        \lim_{t \rightarrow +\infty} \ee[S(t)] = \lim_{t \rightarrow +\infty} e^{-\lambda t}s = 0
      \end{align*}
      This means that our stock price will on average go to zero in the long run.
    \end{solution}
  \item Does the variance of $S(t)$ increase or decrease over time?
    \begin{solution}
      Variance increases over time as the negative term vanishes with time.
    \end{solution}
\end{enumerate}

\noindent{ \bf Problem 4:} Let $W(t)$ be a brownian motion, and define
  \begin{equation*}
    B(t) = \int^t_0 sign(W(s)) dW(s)
  \end{equation*}
  where
  \begin{align*}
    sign(x) =
      \begin{cases}
        1 &, if x \geq 0 \\
        -1 &, otherwise
      \end{cases}
  \end{align*}

  \begin{enumerate}
    \item Show that $B(t)$ is a Brownian motion
      \begin{solution}
        We check the following:
          \begin{enumerate}
            \item $B(0) = \int^0_0 sign(W(0)) dW(s) = 0$
            \item For $\epsilon > 0$, $B(t + \epsilon) - B(t) = \int^{t+\epsilon}_{t} sign(W(s)) dW(s)$ is independent of increments $B(m) - B(n)$ for
              $0 \leq n \leq m < t$.
            \item Almost-sure continuity follows from the construction of the Ito integral. As in the construction, we can find simple processes such that their integrals are continuous
              and converge in the mean square sense to our integral.
            \item For $\epsilon > 0$, $B(t + \epsilon) - B(t) = \int^{t+\epsilon}_{t} sign(W(s)) dW(s)$. Note that here if we take a partition $\mathcal{P}$ of
              $t = t_1 < t_2 < \cdots < t_n = t+\epsilon$ where the mesh of $\mathcal{P}$ converges to zero, we can approximate the above integral as the sum
              $B(t + \epsilon) - B(t) = \sum_{j=1}^{n} sign(W(t_j)) (W(t_{j}) - W(t_{j-1}))$ so we are considering the sum of positive and negative increments of the wiener process which are
              normal with mean 0 and variance $t_{j} - t_{j-1}$. We note that this normal distribution has mean 0 and variance $\sum_{j=1}^{n} t_{j}-t_{j-1} = t_n - t_1 = \epsilon$.
              So $B(t)$ has increments which are normally distributed with mean zero and variance equal to the increment length.
          \end{enumerate}
      \end{solution}
    \item Use Ito's product rule to compute $d[B(t)W(t)]$. Integrate both sides of the
      resulting equation, take expectations and deduce that
        \begin{equation*}
          \ee[B(t)W(t)] = 0
        \end{equation*}
      \begin{solution}
        We first note that $dB(t) = B(t+dt) - B(t) = \int^{t+dt}_t sign(W(s)) dW(s) = sign(W(t))dW(t)$
        \begin{align*}
          d[B(t)W(t)] = sign(W(t))dW(t)W(t) + B(t)dW(t) + sign(W(t))dW(t)dW(t)
        \end{align*}
        We integrate both sides from $0$ to $t$ to get:
        \begin{align*}
          B(t)W(t) - B(0)W(0) &= \int^t_0 d[B(s)W(s)] ds \\
          &= \int^t_0 W(s)dB(s) + \int^t_0 B(s) dW(s) + \int^t_0 dW(s)dB(s)
        \end{align*}
        We take expectation to get:
        \begin{align*}
          \ee[B(t)W(t)] &= \ee[\int^t_0 W(s)sign(W(s)) dW(s) + \int^t_0 B(s) dW(s) + \int^t_0 sign(W(s))dW(s)dW(s)] \\
          &= \ee[\int^t_0 W(s)sign(W(s)) dW(s)] + \ee[\int^t_0 B(s) dW(s)] + \ee[\int^t_0 sign(W(s)) ds] \\
          &= 0 + 0 + \ee[\int^t_0 sign(W(s)) ds] \\
          &= \int^t_0 \ee[sign(W(s))] ds \\
          &= \int^t_0 \big(1\cdot \frac{1}{2} + (-1) \cdot \frac{1}{2} \big) ds = 0
        \end{align*}
      \end{solution}
    \item Verify that
      \begin{equation*}
        dW^2(t) = 2W(t)dW(t) + dt
      \end{equation*}
      \begin{solution}
        \begin{align*}
          d(W^2(t)) = dW(t)W(t) + W(t)dW(t) + dW(t)dW(t) = 2W(t)dW(t) + dt
        \end{align*}
      \end{solution}
    \item Use Ito product's rule to $d[B(t)W^2(t)]$. Deduce that
      \begin{equation*}
        \ee[B(t)W^2(t)] \text{ is not equal to } \ee[B(t)]\ee[W^2(t)]
      \end{equation*}
      \begin{solution}
        \begin{align*}
          &d[B(t)W^2(t)] \\
          &= dB(t)W^2(t) + B(t)d(W^2(t)) + dB(t)d(W^2(t))\\
          &= dB(t)W^2(t) + B(t)(2W(t)dW(t) + dt) + dB(t)(2W(t)dW(t) + dt) \\
          &=  \bigg(sign(W(t))W^2(t) + B(t)2W(t)\bigg) dW(t) + B(t)dt + sign(W(t))dW(t)dt + \\
          &+ sign(W(t))dW(t)2W(t)dW(t) \\
          &=  \bigg(sign(W(t))W^2(t) + B(t)2W(t)\bigg) dW(t) + B(t)dt + 0 + sign(W(t))2W(t)dt \\
          &= \bigg(sign(W(t))W^2(t) + B(t)2W(t)\bigg) dW(t) + \bigg(B(t) + sign(W(t))2W(t)\bigg) dt \\
        \end{align*}
        We take integration over 0 to $t$ and expectation to get:
        \begin{align*}
          &\ee[B(t)W^2(t)] \\
          &= B(0)W^2(0) + \ee[\int^t_0 \bigg(sign(W(s))W^2(s) +B(s)2W(s)\bigg) dW(s) + \\
          &+ \int^t_0 \bigg(B(s) + sign(W(s))2W(s)\bigg) ds] \\
          &= 0 + \ee[\int^t_0 \bigg(B(s) + sign(W(s))2W(s)\bigg) ds] \\
          &= \int^t_0 \ee[B(s) + sign(W(s))2W(s)] ds \\
          &= \int^t_0 \ee[sign(W(s))2W(s)] ds \\
          &= 2\int^t_0 \ee[|W(s)|] ds
        \end{align*}
        We note that $\ee[B(t)]\ee[W^2(t)] = 0$ so that if $\ee[B(t)W^2(t)] = \ee[B(t)]\ee[W^2(t)]$,
        $2 \int^t_0 \ee[|W(s)|] ds = 0 \Rightarrow W(s) = 0 \forall s \in [0,t]$. But this is evidently not true.
        It then follows that:
        \begin{equation*}
          \ee[B(t)W^2(t)] \text{ is not equal to } \ee[B(t)]\ee[W^2(t)]
        \end{equation*}
      \end{solution}
    \item Conclude
      \begin{solution}
        We can conclude that $B(t)$ and $W^2(t)$ are not independent random processes.
      \end{solution}
  \end{enumerate}

\noindent{\bf Problem 5}
  \par{In the Hull-white interest rate model, the interest rate is given by the stochastic
  differential equation}
  \begin{equation*}
    dR(t) = (a(t) - b(t)R(t))dt + \sigma(t) d\tilde{W}(t)
  \end{equation*}
  where $a(t), b(t), R(t), \sigma(t)$ are nonrandom positive functions and $\tilde{W}$ is a
  standard Brownian motion under a risk-neutral probability measure. Assume that the initial
  condition is given at time $t$ by $R(t) = r$. The goal of this exercise is to solve this
  equation explicitly.

  \begin{enumerate}
    \item First of all, compute, by using Ito-Doeblin formula
      \begin{equation*}
        d(e^{\int^u_0 b(v) dv}R(u))
      \end{equation*}
      \begin{solution}
        As $b(t)$ is nonrandom, we don't need the Ito-Doeblin formula and proceed with standard product rule:
        \begin{align*}
          d(e^{\int^u_0 b(v) dv}R(u)) = d(e^{\int^u_0 b(v) dv})R(u) + e^{\int^u_0 b(v) dv}d(R(u)) \\
          = b(u)e^{\int^u_0 b(v) dv} R(u) du + e^{\int^u_0 b(v) dv}((a(u) - b(u)R(u))du + \sigma(u) d\tilde{W}(u)) \\
          = e^{\int^u_0 b(v) dv}a(u)du + e^{\int^u_0 b(v) dv}\sigma(u) d\tilde{W}(u)
        \end{align*}
      \end{solution}
    \item Integrate both sides of the formula found above from $t$ to $T$ and use the initial
      condition $R(t) = r$ to find the formula
      \begin{equation*}
        e^{\int^T_0 b(v) dv} R(T) = re^{\int^u_0 b(v)dv} + \int^T_0 e^{\int^u_0 b(v)dv} a(u)du
        + \int^T_0 e^{\int^t_0 b(v)dv} \sigma(u) d\tilde{W}
      \end{equation*}
      \begin{solution}
        We integrate both sides from $t$ to $T$ for:
        \begin{align*}
          e^{\int^T_0 b(v) dv}R(T) - e^{\int^t_0 b(v) dv}R(t) = \int^T_t e^{\int^u_0 b(v) dv}a(u)du + \int^T_t e^{\int^u_0 b(v) dv}\sigma(u) d\tilde{W}(u) \\
          e^{\int^T_0 b(v) dv}R(T) = re^{\int^t_0 b(v) dv} + \int^T_t e^{\int^u_0 b(v) dv}a(u)du + \int^T_t e^{\int^u_0 b(v) dv}\sigma(u) d\tilde{W}(u)
        \end{align*}
      \end{solution}
    \item Solve the above formula for $R(T)$. This yields an explicit formula for $R(T)$
      \begin{solution}
        \begin{align*}
          R(T) &= \frac{re^{\int^t_0 b(v) dv} + \int^T_t e^{\int^u_0 b(v) dv}a(u)du + \int^T_t e^{\int^u_0 b(v) dv}\sigma(u) d\tilde{W}(u))}{e^{\int^T_0 b(v) dv}} \\
          &= \frac{re^{\int^t_0 b(v) dv} + \int^T_t e^{\int^u_0 b(v) dv}a(u)du + \int^T_t e^{\int^u_0 b(v) dv}\sigma(u) d\tilde{W}(u))}{e^{\int^T_0 b(v) dv}} \\
          &= re^{-\int^T_t b(v) dv} + \int^T_t e^{-\int^T_u b(v) dv}a(u)du + \int^T_t e^{-\int^T_u b(v) dv}\sigma(u) d\tilde{W}(u))
        \end{align*}
      \end{solution}
    \item Give the distribution of $R(T)$
      \begin{solution}
        We note that as $b(t), a(t), \sigma(t)$ are nonrandom functions, $re^{-\int^T_t b(v) dv}$ and $\int^T_t e^{-\int^T_u b(v) dv}a(u)du$ are constants. We then note that
        the randomness of $R(T)$ comes from $\int^T_t e^{-\int^T_u b(v) dv}\sigma(u) d\tilde{W}(u) = \sum^n_{j=1} e^{-\int^T_{t_{j}} b(v) dv} \sigma(t_{j})\cdot (W(t_{j}) - W(t_{j-1}))$.
        Where the sequence $t = t_1 < t_2 < \cdots < t_n = T$ represents a partition of $[t, T]$ with mesh converging to zero.
        It then follows that $R(T)$ follows a normal distribution as the sum of separately scaled independent wiener increments which are normal.
      \end{solution}
    \item Give also its mean and variance.
      \begin{solution}
        We first observe that the mean is:
        \begin{align*}
          \ee[R(T)] &= \ee[re^{-\int^T_t b(v) dv} + \int^T_t e^{-\int^T_u b(v) dv}a(u)du + \int^T_t e^{-\int^T_u b(v) dv}\sigma(u) d\tilde{W}(u)] \\
          &= re^{-\int^T_t b(v) dv} + \int^T_t e^{-\int^T_u b(v) dv}a(u)du + \ee[\int^T_t e^{-\int^T_u b(v) dv}\sigma(u) d\tilde{W}(u)] \\
          &= re^{-\int^T_t b(v) dv} + \int^T_t e^{-\int^T_u b(v) dv}a(u)du + 0 \\
          &= re^{-\int^T_t b(v) dv} + \int^T_t e^{-\int^T_u b(v) dv}a(u)du
        \end{align*}
        And that the variance is:
        \begin{align*}
          Var[R(T)] &=  Var[re^{-\int^T_t b(v) dv} + \int^T_t e^{-\int^T_u b(v) dv}a(u)du + \int^T_t e^{-\int^T_u b(v) dv}\sigma(u) d\tilde{W}(u)] \\
          &= Var[\int^T_t e^{-\int^T_u b(v) dv}\sigma(u) d\tilde{W}(u)] \\
          &= \ee[\big(\int^T_t e^{-\int^T_u b(v) dv}\sigma(u) d\tilde{W}(u)\big)^2] - \bigg(\ee[\int^T_t e^{-\int^T_u b(v) dv}\sigma(u) d\tilde{W}(u)]\bigg)^2 \\
          &= \ee[\int^T_t \big(e^{-\int^T_u b(v) dv}\sigma(u)\big)^2 du] - 0 \\
          &= \ee[\int^T_t e^{-2\int^T_u b(v) dv}\sigma^2(u) du] \\
          &= \int^T_t e^{-2\int^T_u b(v) dv}\sigma^2(u) du
        \end{align*}
      \end{solution}
  \end{enumerate}


\end{document}
