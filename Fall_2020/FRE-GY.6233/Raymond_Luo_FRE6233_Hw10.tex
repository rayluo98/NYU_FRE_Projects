\documentclass[12pt,twoside, letter]{exam}
\usepackage{enumitem, kantlipsum}
\usepackage[margin=1in,left=1in,right=1in,top=1in,bottom=1in]{geometry}
\usepackage{graphicx,epstopdf}
\usepackage{amssymb,amsmath,amsfonts, amsthm}
\usepackage{wasysym}
\newtheorem{theorem}{Theorem}
\newtheorem{corollary}{Corollary}[theorem]
\newtheorem{lemma}[theorem]{Lemma}
\usepackage{hyperref}
\usepackage{tikz}
\usepackage{xstring}
\usetikzlibrary{calc}
\usepackage{ducksay}
\newtheorem{prop}{Proposition}

\theoremstyle{definition}
\newtheorem{definition}{Definition}

\usepackage{bbm}
\usepackage{verbatim}
\usepackage{bbold}
\usepackage{phfparen}
\usepackage{sets}
\newcommand{\nn}{\mathbb{N}}
\newcommand{\rr}{\mathbb{R}}
\newcommand{\cc}{\mathbb{C}}
\newcommand{\cb}{\mathcal{B}}
\newcommand{\ctau}{\mathcal{T}}
\newcommand{\co}{\mathcal{O}}
\newcommand{\zz}{\mathbb{Z}}
\newcommand{\ee}{\mathbb{E}}
\newcommand{\qq}{\mathbb{Q}}
\newcommand{\interior}{\text{Int}}
\newcommand{\pp}{\mathbb{P}}
\newcommand{\id}{\mathbbm{1}}
\newcommand{\Co}{\text{Co}}
\newcommand{\Cl}{\text{Cl}}
\usepackage{mathtools}
\DeclarePairedDelimiter\ceil{\lceil}{\rceil}
\DeclarePairedDelimiter\floor{\lfloor}{\rfloor}


\usepackage{indentfirst}
\setlist{
    listparindent = \parindent,
    parsep = 6pt,
}

\makeatletter
\newsavebox\myboxA
\newsavebox\myboxB
\newlength\mylenA

\newcommand*\xoverline[2][0.9]{%
    \sbox{\myboxA}{$\m@th#2$}%
    \setbox\myboxB\null% Phantom box
    \ht\myboxB=\ht\myboxA%
    \dp\myboxB=\dp\myboxA%
    \wd\myboxB=#1\wd\myboxA% Scale phantom
    \sbox\myboxB{$\m@th\overline{\copy\myboxB}$}%  Overlined phantom
    \setlength\mylenA{\the\wd\myboxA}%   calc width diff
    \addtolength\mylenA{-\the\wd\myboxB}%
    \ifdim\wd\myboxB<\wd\myboxA%
       \rlap{\hskip 0.5\mylenA\usebox\myboxB}{\usebox\myboxA}%
    \else
        \hskip -0.5\mylenA\rlap{\usebox\myboxA}{\hskip 0.5\mylenA\usebox\myboxB}%
    \fi}
\makeatother

\usepackage{float}
\floatstyle{boxed}
\restylefloat{figure}

\printanswers

\begin{document}


\abovedisplayskip=12pt
\belowdisplayskip=12pt
\abovedisplayshortskip=7pt
\belowdisplayshortskip=10pt
\allowdisplaybreaks

\setlength{\parindent}{18pt}

\title{FRE-GY 6233: Assignment 10}
\author{Raymond Luo}
\date{\today}
\maketitle

\noindent{\bf Problem 1}
\par{Let your initial capital $x = 200$, the total capital $A = 400$, and the probability (for "you") to win a
dollar $p = \frac{31}{63}$. Suppose you bet 1,2,5,10,20,40 dollars each time. For each case calculate:}

\begin{itemize}
  \item The probability of winning before you lose everything
    \begin{solution}
      We note that for $p \neq q := \pp['Loss']$, we can deduce a closed form solution for winning
      based on the system of equations:
      \begin{align*}
        \begin{cases}
          u(x) = pu(x+1) + qu(x-1) \\
          u(0) = 1, u(A) = 0
        \end{cases}
      \end{align*}
        By the difference equation $u(x) = C_1 + C_2(q/p)^x$ to get that:
        \begin{equation*}
          u(x) = \frac{(q/p)^A - (q/p)^x}{(q/p)^A-1}
        \end{equation*}
        From this, we have that:
        \begin{itemize}
          \item If we bet 1 dollars: \\
            This is the same as plugging directly into the formula to find that:
            \begin{align*}
              \pp['Win' \mid X_0 = 200] &= 1 - u(200) = 1 - \frac{(q/p)^A - (q/p)^x}{(q/p)^A-1} \\
              &= \frac{(32/31)^{400} - (32/31)^{200}}{(32/31)^{400}-1} = 0.001744
            \end{align*}
          \item If we bet 2 dollars: \\
            This is effectively the same setup with $A' = A/2 = 200$,
            $X_0'= X_0/2 = 100$ to find that:
            \begin{align*}
              \pp['Win' \mid X_0' = 100] &= 1 - u(100) = 1 - \frac{(q/p)^A - (q/p)^x}{(q/p)^A-1} \\
              &= \frac{(32/31)^{200} - (32/31)^{100}}{(32/31)^{200}-1} = 0.04012
            \end{align*}
          \item If we bet 5 dollars: \\
            This is effectively the same setup with $A' = A/5 = 80$,
            $X_0'= X_0/5 = 40$ to find that:
            \begin{align*}
              \pp['Win' \mid X_0' = 40] &= 1 - u(40) = 1 - \frac{(q/p)^A - (q/p)^x}{(q/p)^A-1} \\
              &= \frac{(32/31)^{80} - (32/31)^{40}}{(32/31)^{80}-1} = 0.21927
            \end{align*}
            \item If we bet 10 dollars: \\
              This is effectively the same setup with $A' = A/10 = 40$,
              $X_0'= X_0/10 = 20$ to find that:
              \begin{align*}
                \pp['Win' \mid X_0' = 20] &= 1 - u(20) = 1 - \frac{(q/p)^A - (q/p)^x}{(q/p)^A-1} \\
                &= \frac{(32/31)^{40} - (32/31)^{20}}{(32/31)^{40}-1} = 0.34638
              \end{align*}
            \item If we bet 20 dollars: \\
              This is effectively the same setup with $A' = A/20 = 20$,
              $X_0'= X_0/20 = 10$ to find that:
              \begin{align*}
                \pp['Win' \mid X_0' = 10] &= 1 - u(10) = 1 - \frac{(q/p)^A - (q/p)^x}{(q/p)^A-1} \\
                &= \frac{(32/31)^{20} - (32/31)^{10}}{(32/31)^{20}-1} = 0.42129
              \end{align*}
            \item If we bet 40 dollars: \\
              This is effectively the same setup with $A' = A/40 = 10$,
              $X_0'= X_0/40 = 5$ to find that:
              \begin{align*}
                \pp['Win' \mid X_0' = 5]& = u(5) = 1 - \frac{(q/p)^A - (q/p)^x}{(q/p)^A-1} \\
                &= \frac{(32/31)^{10} - (32/31)^5}{(32/31)^{10}-1} = 0.46040
              \end{align*}
        \end{itemize}
    \end{solution}
  \item The expected duration of the game.
    \begin{solution}
      We note that for $p \neq q := \pp['Loss']$, we can deduce a closed form solution for winning
      based on the system of equations:
      \begin{align*}
        \begin{cases}
          D(x) = pD(x+1) + qD(x-1) + 1 \\
          D(0) = 0, D(A) = 0
        \end{cases}
      \end{align*}
        We solve inhomogeneous equation to get:
        \begin{equation*}
          D(x) = \frac{x}{q-p} + \frac{A}{q-p}\frac{1-(q/p)^x}{1-(q/p)^A}
        \end{equation*}
        From this, we have that:
        \begin{itemize}
          \item If we bet 1 dollars: \\
            This is the same as plugging directly into the formula to find that:
            \begin{align*}
              D(200) &= \frac{x}{q-p} + \frac{A}{q-p}\frac{1-(q/p)^x}{1-(q/p)^A} \\
              &= 200\cdot 63 + 400\cdot 63 \cdot \frac{1-(32/31)^{200}}{1-(32/31)^{400}} = 12643.95
            \end{align*}
          \item If we bet 2 dollars: \\
            This is effectively the same setup with $A' = A/2 = 200$,
            $X_0'= X_0/2 = 100$ to find that:
            \begin{align*}
              D(100) &= \frac{x}{q-p} + \frac{A}{q-p}\frac{1-(q/p)^x}{1-(q/p)^A} \\
              &= 100\cdot 63 + 200\cdot 63 \cdot \frac{1-(32/31)^{100}}{1-(32/31)^{200}} = 6805.54
            \end{align*}
          \item If we bet 5 dollars: \\
            This is effectively the same setup with $A' = A/5 = 80$,
            $X_0'= X_0/5 = 40$ to find that:
            \begin{align*}
              D(40) &= \frac{x}{q-p} + \frac{A}{q-p}\frac{1-(q/p)^x}{1-(q/p)^A} \\
              &= 40\cdot 63 + 80\cdot 63 \cdot \frac{1-(32/31)^{40}}{1-(32/31)^{80}} = 3625.101
            \end{align*}
            \item If we bet 10 dollars: \\
              This is effectively the same setup with $A' = A/10 = 40$,
              $X_0'= X_0/10 = 20$ to find that:
              \begin{align*}
                D(20) &= \frac{x}{q-p} + \frac{A}{q-p}\frac{1-(q/p)^x}{1-(q/p)^A} \\
                &= 20\cdot 63 + 40\cdot 63 \cdot \frac{1-(32/31)^{20}}{1-(32/31)^{40}} = 2132.887
              \end{align*}
            \item If we bet 20 dollars: \\
              This is effectively the same setup with $A' = A/20 = 20$,
              $X_0'= X_0/20 = 10$ to find that:
              \begin{align*}
                D(10) &= \frac{x}{q-p} + \frac{A}{q-p}\frac{1-(q/p)^x}{1-(q/p)^A} \\
                &= 10\cdot 63 + 20\cdot 63 \cdot \frac{1-(32/31)^{10}}{1-(32/31)^{20}} = 1160.82
              \end{align*}
            \item If we bet 40 dollars: \\
              This is effectively the same setup with $A' = A/40 = 10$,
              $X_0'= X_0/40 = 5$ to find that:
              \begin{align*}
                D(5) &= \frac{x}{q-p} + \frac{A}{q-p}\frac{1-(q/p)^x}{1-(q/p)^A} \\
                &= 5\cdot 63 + 10\cdot 63 \cdot \frac{1-(32/31)^{5}}{1-(32/31)^{10}} = 605.05
              \end{align*}
        \end{itemize}
    \end{solution}
\end{itemize}
Which strategy would you choose?
  \begin{solution}
    It appears that the best strategy is to bet with larger denominations of money for the highest chance of winning.
  \end{solution}

\noindent{\bf Problem 2}
\par{Consider the same problem as in class, down-and-out call with barrier $X$ below strike $K$ and $S > K$.}

\begin{enumerate}
  \item Derive the formula by solving a PDE problem with a slight modification. Instead of change of variable
    \begin{equation*}
      S = Ke^{x}
    \end{equation*}
    use a change of variable
    \begin{equation*}
      S = Xe^{x}
    \end{equation*}
    Show all details
    \begin{solution}
      We use change of variables by $S = Xe^x, t = T - \frac{\tau}{\frac{1}{2}\sigma^2}, c(t,S) = Kv(x, \tau)$:
      \begin{align*}
        c(t,S) = Kv(\ln(\frac{S}{X}), \frac{\sigma^2}{2}(T-t))
      \end{align*}
      We then have the following by differentiating:
      \begin{align*}
        \frac{\partial c}{\partial t} = \frac{\partial \tau}{\partial t} \cdot \frac{\partial v}{\partial \tau}
        = \frac{-\sigma^2}{2}K \frac{\partial v}{\partial \tau} \\
        \frac{\partial c}{\partial S} = \frac{\partial x}{\partial S} \cdot \frac{\partial v}{\partial x}
        = \frac{K}{S} \frac{\partial v}{\partial x} \\
        \frac{\partial^2 c}{\partial S^2} = \frac{\partial}{\partial S} \frac{K}{S} \frac{\partial v}{\partial x}
        = \frac{K}{S^2}\frac{\partial^2 v}{\partial x^2} - \frac{K}{S^2} \frac{\partial v}{\partial x}
      \end{align*}
      With $k_1 = \frac{2r}{\sigma^2}$, we get the same PDE as before:
      \begin{equation*}
      \frac{\partial v}{\partial \tau} = \frac{\partial^2 v}{\partial x^2}
      + \bigg(k_1 - 1\bigg)\frac{\partial v}{\partial x} - k_1v
      \end{equation*}
      Note that we can proceed as before with $v(\tau, x) = exp(\alpha x + \beta \tau)u(\tau, x)$ to get
      \begin{align*}
        LHS &= exp(\alpha x + \beta \tau)\bigg(\beta u + \frac{\partial u}{\partial \tau} \bigg) \\
        RHS &= exp(\alpha x + \beta \tau)\bigg[(\alpha^2u + \frac{\partial^2 v}{\partial x^2}) + (k_1 - 1)
        \big(\alpha u + \frac{\partial u}{\partial x} \big) - k_1 u
        \big)\bigg] \\
        \beta u + \frac{\partial u}{\partial \tau} &= \alpha^2u + \frac{\partial^2 v}{\partial x^2} + (k_1 - 1)
        \big(\alpha u + \frac{\partial u}{\partial x} \big) - k_1 u
      \end{align*}
        Note that we choose $\alpha = \frac{-1}{2}(k_1 - 1)$, which means that from (1), $\beta = \frac{-1}{4}(k_1 + 1)^2$.
        Plugging this into our change of variable equation, we have that:\\
        $v(\tau, x) = exp(\frac{-1}{2}(k_1 - 1) x + \frac{-1}{4}(k_1 + 1)^2 \tau)u(\tau, x)$ so that we have differential equation: \\
        $\frac{\partial u}{\partial \tau} = \frac{\partial^2 v}{\partial x^2}$ with boundary condition \\
        $u(x,0) = e^{-\alpha x}\max\bigg(e^{x} - 1, 0 \bigg)
        = \max\bigg(e^{\frac{1}{2}(k_1 + 1)x} - e^{\frac{1}{2}(k_1 - 1) x}, 0 \bigg)$
        with an additional condition that $u(log(X/K)) = 0$ \\
        Here we use the method of images to solve for a closed-form solution of the PDE. We reflect the initial data about $x_0 = \ln(X/K)$
        to get equations:
        \begin{align*}
          u(x,0) &= u_0(x)(2x_0 - x) \\
          u(x,0) &=
            \begin{cases}
              \max{\bigg(e^{\frac{1}{2}(k_1 + 1)x} - e^{\frac{1}{2}(k_1 - 1)x}, 0 \bigg)} & \text{ for } x > x_0 \\
              -\max{\bigg(e^{\frac{1}{2}(k_1 + 1)(x_0 - \frac{1}{2}x)} - e^{\frac{1}{2}(k_1 - 1)(x_0 - \frac{1}{2}x)}, 0 \bigg)} & \text{ for } x < x_0
            \end{cases}
        \end{align*}
        We note that the value of call option $C(S,t) = Ke^{\alpha x + \beta \tau}u_1(x,\tau)$ and that $V(S,t)= Ke^{\alpha x + \beta \tau}(u_1(x,\tau)
        + u_2(x,\tau))$ where $u_2(x,\tau) = -u_1(2x_0 - x, \tau)$. We then get that
        $V(S,t) = C(S,t)- (\frac{S}{X})^{-(k_1 - 1)} C(X^2/S,t)$
    \end{solution}
  \item Does this formula work in the case $X > K$?
    \begin{solution}
      No, our method of images would fail to preserve the Black Scholes Call option pricing for the entirety of
      $X > K$ so our closed form extension would change. Moreover,
      if $X > K$, we effectively have a call option with strike price $X$ instead of $K$ so we wouldn't need to derive the formula in this manner.
    \end{solution}
\end{enumerate}


\noindent{\bf Problem 3}
\par{Formulate the problem for a "down-in" barrier call option: the option is activated and becomes a vanilla call option with strike $K$
only if asset price $S$ crosses a lower barrier $X$ before expiry: the payoff is $\max{(S-K, 0)}$ if $S \leq X$ before expiry and zero otherwise.
Assume $X < S$, $X < K$.

\begin{enumerate}
  \item Write down the formula for this option (look for connection with a "down-out" call barrier with the same barrier, strike, and expiry)
    \begin{solution}
      With the "down-out" call option value denoted $c_{out}$ and "down-in" call option denoted $c_{in}$, we have that:
      \begin{equation*}
        C(s,t) = c_{out}(s,t) + c_{in}(s,t)
      \end{equation*}
      So that we have that:
      \begin{equation*}
        c_{in}(s,t) = C(S,t) - c_{out}(s,t) =  (\frac{S}{X})^{-(k_1 - 1)} C(X^2/S,t)
      \end{equation*}
    \end{solution}
  \item Calculate both "down-in" and "down-out" in the case when the call options are written on \textit{futures}, with the initial price $F = 30,
    K = 40, T = 1$ (years), $\sigma = 0.5, r = 0.01, X = 20$.
      \begin{solution}
        "Down-out" Call option:
        \begin{align*}
          c_{out}(F = 30, t = 0) &= (\frac{30}{20})^{-(\frac{0.02}{0.5^2} - 1)} C(20^2/30,0) \\
          &=  1.4521*0.0584 = \$ 0.0848
        \end{align*}
        "Down-in" Call option:
        \begin{align*}
          c_{in}(F = 30, t = 0) &= C(30, 0)- (\frac{30}{20})^{-(\frac{0.02}{0.5^2} - 1)} C(20^2/30,0) \\
          &= 3.0728 - 0.0848 = \$ 2.988
        \end{align*}
      \end{solution}
  \item When does "down-out" becomes practically a vanilla call?
    \begin{solution}
      A "down-out" call option is practically a vanilla call as long as it does not go below a barrier $X$ at any time prior to maturity.
    \end{solution}
\end{enumerate}


\noindent{\bf Problem 4}
\par{Using the method of images find a formula for an "up-out" put option with barrier $X$, with
payoff $max(K - F, 0)$ if $F < X$ before expiry $T$ and zero otherwise. Assume $X > K, F < X$.
Price it in the case option is written on futures, $F = 30, K = 15, X = 40, \sigma = 0.6, T = 0.5,
r = 0.01$.}
  \begin{solution}
    We begin by noting that the put option has the same PDE as our call option just with modified terminal condition. \\
    We have differential equation: \\
        $\frac{\partial u}{\partial \tau} = \frac{\partial^2 v}{\partial x^2}$ with boundary condition \\
        $u(x,0) = e^{-\alpha x}\max\bigg(1 - e^{x}, 0 \bigg)
        = \max\bigg(e^{\frac{1}{2}(k_1 - 1) x} - e^{\frac{1}{2}(k_1 + 1)x}, 0 \bigg)$
        with an additional condition that $u(log(X/K)) = 0$ \\
    Here we use the method of images to solve for a closed-form solution of the PDE. We reflect the initial data about $x_0 = \ln(X/K)$
    to get equations:
    \begin{align*}
      u(x,0) &= u_0(x)(2x_0 - x) \\
      u(x,0) &=
        \begin{cases}
           \max\bigg(e^{\frac{1}{2}(k_1 - 1) x} - e^{\frac{1}{2}(k_1 + 1)x}, 0 \bigg) & \text{ for } x > x_0 \\
          -\max{\bigg(e^{\frac{1}{2}(k_1 - 1)(x_0 - \frac{1}{2}x)} - e^{\frac{1}{2}(k_1 + 1)(x_0 - \frac{1}{2}x)}, 0 \bigg)} & \text{ for } x < x_0
        \end{cases}
    \end{align*}
    This proceeds as before except now we observe that the value of put option $P(S,t) = Ke^{\alpha x + \beta \tau}u_1(x,\tau)$
    and that $V(S,t)= Ke^{\alpha x + \beta \tau}(u_1(x,\tau)
    + u_2(x,\tau))$ where $u_2(x,\tau) = -u_1(2x_0 - x, \tau)$. We then get that
    $V(S,t) = P(S,t)- (\frac{S}{X})^{-(k_1 - 1)} P(X^2/S,t)$ \\
    The "up-out" put option is price at:
      \begin{align*}
        V(S,T) &= V(30, 0.5) = P(30, 0.5) - (30/40)^{1-0.02/0.6^2}P(40^2/30, 0.5) \\
        &= 0.1841 - 0.0034 =  0.1807
      \end{align*}
  \end{solution}


\end{document}
