\documentclass[12pt,twoside, letter]{exam}
\usepackage{enumitem, kantlipsum}
\usepackage[margin=1in,left=1in,right=1in,top=1in,bottom=1in]{geometry}
\usepackage{graphicx,epstopdf}
\usepackage{amssymb,amsmath,amsfonts, amsthm}
\usepackage{wasysym}
\newtheorem{theorem}{Theorem}
\newtheorem{corollary}{Corollary}[theorem]
\newtheorem{lemma}[theorem]{Lemma}
\usepackage{hyperref}
\usepackage{tikz}
\usepackage{xstring}
\usetikzlibrary{calc}
\usepackage{ducksay}
\newtheorem{prop}{Proposition}

\theoremstyle{definition}
\newtheorem{definition}{Definition}

\usepackage{bbm}
\usepackage{verbatim}
\usepackage{bbold}
\usepackage{phfparen}
\usepackage{sets}
\newcommand{\nn}{\mathbb{N}}
\newcommand{\rr}{\mathbb{R}}
\newcommand{\cc}{\mathbb{C}}
\newcommand{\cb}{\mathcal{B}}
\newcommand{\ctau}{\mathcal{T}}
\newcommand{\co}{\mathcal{O}}
\newcommand{\zz}{\mathbb{Z}}
\newcommand{\ee}{\mathbb{E}}
\newcommand{\qq}{\mathbb{Q}}
\newcommand{\interior}{\text{Int}}
\newcommand{\pp}{\mathbb{P}}
\newcommand{\id}{\mathbbm{1}}
\newcommand{\Co}{\text{Co}}
\newcommand{\Cl}{\text{Cl}}
\usepackage{mathtools}
\DeclarePairedDelimiter\ceil{\lceil}{\rceil}
\DeclarePairedDelimiter\floor{\lfloor}{\rfloor}


\usepackage{indentfirst}
\setlist{
    listparindent = \parindent,
    parsep = 6pt,
}

\makeatletter
\newsavebox\myboxA
\newsavebox\myboxB
\newlength\mylenA

\newcommand*\xoverline[2][0.9]{%
    \sbox{\myboxA}{$\m@th#2$}%
    \setbox\myboxB\null% Phantom box
    \ht\myboxB=\ht\myboxA%
    \dp\myboxB=\dp\myboxA%
    \wd\myboxB=#1\wd\myboxA% Scale phantom
    \sbox\myboxB{$\m@th\overline{\copy\myboxB}$}%  Overlined phantom
    \setlength\mylenA{\the\wd\myboxA}%   calc width diff
    \addtolength\mylenA{-\the\wd\myboxB}%
    \ifdim\wd\myboxB<\wd\myboxA%
       \rlap{\hskip 0.5\mylenA\usebox\myboxB}{\usebox\myboxA}%
    \else
        \hskip -0.5\mylenA\rlap{\usebox\myboxA}{\hskip 0.5\mylenA\usebox\myboxB}%
    \fi}
\makeatother

\usepackage{float}
\floatstyle{boxed}
\restylefloat{figure}

\printanswers

\begin{document}


\abovedisplayskip=12pt
\belowdisplayskip=12pt
\abovedisplayshortskip=7pt
\belowdisplayshortskip=10pt
\allowdisplaybreaks

\setlength{\parindent}{18pt}

\title{FRE-GY 6233: Assignment 9}
\author{Raymond Luo}
\date{\today}
\maketitle

\noindent{\bf Problem 1}
  \begin{itemize}
    \item We assume that over each of the next three years the stock price
      either moves up by 20 percent or moves down by 20 percent .
      The risk-free interest rate is 5 percent
    \item Find the value of a 2-year American put with a strike price of 49
      on a stock whose current price is 50.
  \end{itemize}

  \begin{solution}
    We have the following binomial pricing tree: \\
    \tikzstyle{bag} = [text width=2em, text centered]
    \tikzstyle{end} = []
    \begin{tikzpicture}[sloped]
       \node (a) at ( 0,0) [bag] {$\$ 50$};
       \node (b) at ( 4,-1.5) [bag] {$\$40$};
       \node (c) at ( 4,1.5) [bag] {$\$60$};
       \node (d) at ( 8,-3) [bag] {$\$32$};
       \node (e) at ( 8,0) [bag] {$\$48$};
       \node (f) at ( 8,3) [bag] {$\$72$};
       \draw [->] (a) to node [below] {$(1-p)$} (b);
       \draw [->] (a) to node [below] {$P$} (c);
       \draw [->] (c) to node [below] {$P^2$} (f);
       \draw [->] (c) to node [below] {$(1-p)p$} (e);
       \draw [->] (b) to node [below] {$(1-p)p$} (e);
       \draw [->] (b) to node [below] {$(1-p)^2$} (d);
    \end{tikzpicture}
    \\
    We find that the risk-neutral measure $\mathbb{P}$ for the stock is defined by
    $p = \pp[S_{n+1} = 1.2\cdot S_n] = \frac{1.05 - 0.8}{1.2 - 0.8} = \frac{5}{8}$ \\
    If we denote that the value of the option at time $t$, price $S(t) = s$ as $v(t,s)$,
    we have that:
    \begin{align*}
      v(2,72) &= (49- 72)^+ = \$ 0 \\
      v(2,48) &= (49 - 48)^+ = \$ 1 \\
      v(2,32) &= (49 - 32)^+ = \$ 17 \\
      v(1, 60) &= \max\{\frac{1}{1+r}\big(p\cdot v(2,72) + (1-p) \cdot v(2,48)\big), (49 - 60)^+ \}
      = \$ 0.36\\
      v(1, 40) &= \max\{\frac{1}{1+r}\big(p\cdot v(2,32) + (1-p) \cdot v(2,48)\big), (49 - 40)^+ \}
      = \$9 \\
      v(0, 50) &= \max\{\frac{1}{1+r}\big(p\cdot v(1,40) + (1-p) \cdot v(1,60)\big), (49 - 50)^+ \}
      = \$ 3.43\\
    \end{align*}
  \end{solution}

\noindent{\bf Problem 2}
  \par{Derive the relationship (inequality) between American call $C$
    and put $P$ on stock with price $S$, strike $K$ (no dividends)}
      \begin{solution}
        We first note that we have:
        \begin{align*}
          &C_A - P_A \\
          &=C_E - P_E + P_E - P_A \\
          &=S_0 - Ke^{-r(T-t)} + P_E - P_A \\
          &\leq S_0 - Ke^{-r(T-t)}
        \end{align*}
        We can derive a left-side inequality with the following arbitrage argument: \\
        \begin{itemize}
          \item At time $0$, we long a call option, short a put option, short the underlying stock, and borrow $K$ cash for
           $-C_A + P_A + S_0 - K$.
        \end{itemize}
        In order to not have an arbitrage situation a maturity, we need this value to be less than 0 so that an arbitrage-free requirement is that
        $S(0) - K \leq C_A - P_A$ so that we get relation:
        \begin{equation*}
          S(0) - K \leq C_A - P_A \leq S(0) - Ke^{-r(T-t)}
        \end{equation*}
      \end{solution}

\noindent{\bf Problem 3}
  \par{Consider notations}
  \begin{equation*}
    \tau_1 \vee \tau_2 = \max\{\tau_1, \tau_2\}, \tau_1 \wedge \tau_2 = \min\{\tau_1, \tau_2 \}
  \end{equation*}
  \par{Let $\tau_1$ and $\tau_2$ be stopping times w.r.t filtration $\mathcal{F}_t$. Show that
  the following are also stopping times.}
  \begin{enumerate}
    \item $\tau_1 \vee \tau_2$
      \begin{solution}
        We have that $\{\tau_1 \leq t\}, \{\tau_2 \leq t\} \in \mathcal{F}_t$ for all $t \in T$.
        Note that then: \\
        $\{\tau_1 \vee \tau_2 \leq t \} = \bigcap_{i \in \{1,2\}} \{\tau_{i} \leq t\} \in \mathcal{F}_t$
        for all $t \in T$.
        This naturally belongs to $\mathcal{F}_t$ as the intersection of elements of the sub-sigma-algebra.
      \end{solution}
    \item $\tau_1 \wedge \tau_2$
      \begin{solution}
        We have that $\{\tau_1 \leq t\}, \{\tau_2 \leq t\} \in \mathcal{F}_t$ for all $t \in T$.
        Note that then if we fix $t \in T$: \\
        $\{\tau_1 \wedge \tau_2 \leq t \} = \bigcup_{i \in \{1,2\}} \{\tau_{i} \leq t\}$.
        This naturally belongs to $\mathcal{F}_t$ as the union of elements of the sub-sigma-algebra.
      \end{solution}
    \item $\tau_1 + \tau_2$
      \begin{solution}
        We have that $\{\tau_1 \leq t\}, \{\tau_2 \leq t\} \in \mathcal{F}_t$ for all $t \in T$.
        Note that then if we fix $t \in T$: \\
        $\{\tau_1 + \tau_2 \leq t \} = \bigcup_{r,s \in \qq, r+s < t} \bigg( \{\tau_1 \leq s \} \cap \{\tau_2 \leq r \} \bigg)$.
        This naturally belongs to $\mathcal{F}_t$ as the countable union of elements of the sub-sigma-algebra.
      \end{solution}
    \item $f(\tau), f(t)$ is a continuous increasing function satisfying $f(t) \geq t$
      \begin{solution}
        Note that the condition $f(t) \geq t$ for continuous increasing $f$ is equivalent to $f^{-1}(t) \leq t$
        We have that $\{\tau \leq t\} \in \mathcal{F}_t$ for all $t \in T$.
        Note that then if we fix $t \in T$: \\
        $\{f(\tau) \leq t\} = \{\omega \mid f(\tau(\omega)) \leq t\} = \{\omega \mid \tau(\omega) \leq f^{-1}(t)\}
        \in \mathcal{F}_{f^{-1}(t)}$. \\
        As $f^{-1}(t) \leq t$, $\mathcal{F}_{f^{-1}(t)} \subseteq \mathcal{F}_{t}$ so
        $\{f(\tau) \leq t\} \in \mathcal{F}_{t}$
      \end{solution}
  \end{enumerate}

\noindent{\bf Problem 4}
  \par{An European put option on a non-dividend paying stock and expiring in one year
    is currently selling for 1.8. The stock price is 48 and the strike price is 51
    and the risk free interest rate is 2 percent per year. Can you
    make a riskless profit and how?}
      \begin{solution}
        We first note that $Ke^{-rT} - S = 51\cdot e^{-0.02} - 48= \$ 1.99 > \$1.8 = p$. We have
        an arbitrage opportunity by the following:
        \begin{itemize}
          \item At time $0$, we purchase the put option for $p$ and then exercise immediately for a payoff of $K - S = 51 - 48 = 3$.
            Our total cash flow is $-p + (K - S) = -1.8 + 3 = 1.2 > 0$.
        \end{itemize}
      \end{solution}
\end{document}
