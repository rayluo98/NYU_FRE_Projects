\documentclass[12pt,twoside, letter]{exam}
\usepackage{enumitem, kantlipsum}
\usepackage[margin=1in,left=1in,right=1in,top=1in,bottom=1in]{geometry}
\usepackage{graphicx,epstopdf}
\usepackage{amssymb,amsmath,amsfonts, amsthm}
\usepackage{wasysym}
\newtheorem{theorem}{Theorem}
\newtheorem{corollary}{Corollary}[theorem]
\newtheorem{lemma}[theorem]{Lemma}
\usepackage{hyperref}
\usepackage{tikz}
\usepackage{xstring}
\usetikzlibrary{calc}
\usepackage{ducksay}
\newtheorem{prop}{Proposition}

\theoremstyle{definition}
\newtheorem{definition}{Definition}

\usepackage{bbm}
\usepackage{verbatim}
\usepackage{bbold}
\usepackage{phfparen}
\usepackage{sets}
\newcommand{\nn}{\mathbb{N}}
\newcommand{\rr}{\mathbb{R}}
\newcommand{\cc}{\mathbb{C}}
\newcommand{\cb}{\mathcal{B}}
\newcommand{\ctau}{\mathcal{T}}
\newcommand{\co}{\mathcal{O}}
\newcommand{\zz}{\mathbb{Z}}
\newcommand{\ee}{\mathbb{E}}
\newcommand{\qq}{\mathbb{Q}}
\newcommand{\interior}{\text{Int}}
\newcommand{\pp}{\mathbb{P}}
\newcommand{\id}{\mathbbm{1}}
\newcommand{\Co}{\text{Co}}
\newcommand{\Cl}{\text{Cl}}
\usepackage{mathtools}
\DeclarePairedDelimiter\ceil{\lceil}{\rceil}
\DeclarePairedDelimiter\floor{\lfloor}{\rfloor}


\usepackage{indentfirst}
\setlist{
    listparindent = \parindent,
    parsep = 6pt,
}

\makeatletter
\newsavebox\myboxA
\newsavebox\myboxB
\newlength\mylenA

\newcommand*\xoverline[2][0.9]{%
    \sbox{\myboxA}{$\m@th#2$}%
    \setbox\myboxB\null% Phantom box
    \ht\myboxB=\ht\myboxA%
    \dp\myboxB=\dp\myboxA%
    \wd\myboxB=#1\wd\myboxA% Scale phantom
    \sbox\myboxB{$\m@th\overline{\copy\myboxB}$}%  Overlined phantom
    \setlength\mylenA{\the\wd\myboxA}%   calc width diff
    \addtolength\mylenA{-\the\wd\myboxB}%
    \ifdim\wd\myboxB<\wd\myboxA%
       \rlap{\hskip 0.5\mylenA\usebox\myboxB}{\usebox\myboxA}%
    \else
        \hskip -0.5\mylenA\rlap{\usebox\myboxA}{\hskip 0.5\mylenA\usebox\myboxB}%
    \fi}
\makeatother

\printanswers

\begin{document}

\abovedisplayskip=12pt
\belowdisplayskip=12pt
\abovedisplayshortskip=7pt
\belowdisplayshortskip=10pt
\allowdisplaybreaks

\setlength{\parindent}{18pt}

\title{FRE-GY 6233: Assignment 1}
\author{Raymond Luo}
\date{\today}
\maketitle

\noindent {\bf Problem 1:}
Finish the problem from slide 14 we started in the class.
\begin{enumerate}
  \item List all the sets in $\sigma$-algebra in $\mathcal{F}$
    \begin{solution}
      $\mathcal{F} = \{\emptyset, \Omega, \{a\}, \{b\}, \{c\}, \{d\}, \{a,b\}, \{a,c\}, \{a,d\},$\\
      $\{b,c\}, \{b,d\}, \{c,d\}, \{a,b,c\}, \{b,c,d\}, \{a,c,d\}, \{a,b,d\}, \{a,b,c\} \}$
    \end{solution}
  \item List all the sets $\sigma(X)$
    \begin{solution}
      $\sigma(X) = \{\emptyset, \Omega, \{a,b\}, \{c,d\}\}$
    \end{solution}
  \item Define the r.v. $\ee[X|\sigma(Y))]$, so calculate it for $a,b,c,d$.
    \begin{solution}
      We note that $\pp(a) = \frac{3}{8}, \pp(b) = \frac{1}{8}, \pp(c) = \frac{1}{6}, \pp(d) = \frac{1}{3}$. \\
      As $\ee[Y|X] = \Sigma_{i \in \{-1,1\}} \ee[Y|X = i]\id_{i}(x)$, we consider the following: \\
      If $X = 1$, $\pp(X^{-1}(1) = a) = \frac{\pp(a)}{\pp(a) + \pp(b)} = \frac{3}{4}, \pp(X^{-1}(1) = b) = \frac{1}{4}$\\
      So, $\ee[Y|X = 1] = (-1 \cdot \frac{3}{4} + 1 \cdot \frac{1}{4})
      = -\frac{1}{2}$ \\
      If $X = -1$, $\pp(X^{-1}(-1) = c) = \frac{\pp(c)}{\pp(c) + \pp(d)} = \frac{1}{3}, \pp(X^{-1}(1) = d) = \frac{2}{3}$\\
      So, $\ee[Y|X = -1] = (1 \cdot \frac{1}{3} - 1 \cdot \frac{2}{3})
      = -\frac{1}{3}$
    \end{solution}
\end{enumerate}

\noindent {\bf Problem 2:}
\begin{enumerate}
  \item Suppose that $(\Omega, \mathcal{F}, \mathcal{P})$ is a probability space where $\Omega = \{a,b,c,d,e,f\}$,
    $\mathcal{F}$ is $\sigma$-algebra, and $\mathcal{P}$ is uniform (so $\mathcal{P}(a) = \mathcal{P}(b) = \cdots = \frac{1}{6})$.
  \item Let $X, Y, Z$ be r.v. given by
    \begin{multline*}
      X(a) = 1, X(b) = X(c) = 3, X(d) = X(e) = 5, X(f) = 7 \\
      Y(a) = Y(b) = 2, Y(c) = Y(d) = 1, Y(e) = Y(f) = 7 \\
      Z(a) = Z(b) = Z(c) = Z(d) = 3, Z(e) = Z(f) = 2
    \end{multline*}
\end{enumerate}

\begin{enumerate}
  \item Write down $\sigma(X), \sigma(X), \sigma(Z)$. Are there any relationships between them?
    \begin{solution}
      \begin{multline*}
        \sigma(X) = \{\emptyset, \Omega, \{a\}, \{b,c\}, \{d,e\}, \{f\}, \{a,b,c\}, \{a,d,e\}, \{a,f\}, \\
            \{b,c,d,e\}, \{b,c,f\}, \{d,e,f\},
            \{a,b,c,d,e\}, \{b,c,d,e,f\}, \{a,d,e,f\}, \{a,b,c,f\}\} \\
        \sigma(Y) = \{\emptyset, \Omega, \{a,b\}, \{c,d\}, \{e,f\}, \{a,b,c,d\}, \{a,b,e,f\}, \{c,d,e,f\}\}\\
        \sigma(Z) = \{ \emptyset, \Omega, \{a,b,c,d\}, \{e,f\} \}
      \end{multline*}
    \end{solution}
  \item Define a r.. $\ee[X \mid \sigma(Y)]$
    \begin{solution}
      As $\ee[X|Y] = \Sigma_{i \in \{1,2,7\}} \ee[X|Y = i]\id_{i}(y)$, we consider the following: \\
      If $Y = 1$, $\pp(Y^{-1}(1) = c) = \frac{\pp(c)}{\pp(c) + \pp(d)} = \frac{1}{2}, \pp(Y^{-1}(1) = d) = \frac{1}{2}$\\
      So, $\ee[X|Y = 1] = (X(c) \cdot \frac{1}{2} + X(d) \cdot \frac{1}{2}) = 3\cdot \frac{1}{2} + 5\cdot \frac{1}{2}
      = 4$ \\
      If $Y = 2$, $\pp(X^{-1}(2) = a) = \frac{\pp(a)}{\pp(a) + \pp(b)} = \frac{1}{2}, \pp(X^{-1}(2) = b) = \frac{1}{2}$\\
      So, $\ee[X|Y = 2] = (X(a) \cdot \frac{1}{2} + X(b) \cdot \frac{1}{2}) = 2$ \\
      If $Y = 7$, $\pp(Y^{-1}(7) = e) = \frac{\pp(e)}{\pp(e) + \pp(f)} = \frac{1}{2}, \pp(Y^{-1}(7) = f) = \frac{1}{2}$ \\
      So, $\ee[X|Y = 7] = (X(e) \cdot \frac{1}{2} + X(f) \cdot \frac{1}{2}) = 6$ \\.
    \end{solution}
  \item Check directly the averaging property
    \begin{equation*}
      \ee[\ee[X \mid \sigma(Y)]] = \ee[X]
    \end{equation*}

    \begin{solution}
      $\ee[\ee[X \mid \sigma(Y)]] = \Sigma_{i \in \{1,2,7\}} \ee[X|Y=i] \cdot \pp[Y=i]$. As $\pp[Y=i] = \frac{1}{3}$ for $i \in \{1,2,7\}$, we have that
      $\ee[\ee[X \mid \sigma(Y)]] = \frac{1}{3}\cdot (4 + 2 + 6) = 4$. \\
      We then note that: \\
      $\ee[X] = \Sigma_{i \in \{a,b,c,d,e,f\}} \pp[i] \cdot X(i) = \frac{1}{6}(1+3+3+5+5+7) = \frac{24}{6} = 4$. \\
      From this we have checked the averaging property.
    \end{solution}
\end{enumerate}

\end{document}
