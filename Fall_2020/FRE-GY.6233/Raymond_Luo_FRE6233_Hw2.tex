\documentclass[12pt,twoside, letter]{exam}
\usepackage{enumitem, kantlipsum}
\usepackage[margin=1in,left=1in,right=1in,top=1in,bottom=1in]{geometry}
\usepackage{graphicx,epstopdf}
\usepackage{amssymb,amsmath,amsfonts, amsthm}
\usepackage{wasysym}
\newtheorem{theorem}{Theorem}
\newtheorem{corollary}{Corollary}[theorem]
\newtheorem{lemma}[theorem]{Lemma}
\usepackage{hyperref}
\usepackage{tikz}
\usepackage{xstring}
\usetikzlibrary{calc}
\usepackage{ducksay}
\newtheorem{prop}{Proposition}

\theoremstyle{definition}
\newtheorem{definition}{Definition}

\usepackage{bbm}
\usepackage{verbatim}
\usepackage{bbold}
\usepackage{phfparen}
\usepackage{sets}
\newcommand{\nn}{\mathbb{N}}
\newcommand{\rr}{\mathbb{R}}
\newcommand{\cc}{\mathbb{C}}
\newcommand{\cb}{\mathcal{B}}
\newcommand{\ctau}{\mathcal{T}}
\newcommand{\co}{\mathcal{O}}
\newcommand{\zz}{\mathbb{Z}}
\newcommand{\ee}{\mathbb{E}}
\newcommand{\qq}{\mathbb{Q}}
\newcommand{\interior}{\text{Int}}
\newcommand{\pp}{\mathbb{P}}
\newcommand{\id}{\mathbbm{1}}
\newcommand{\Co}{\text{Co}}
\newcommand{\Cl}{\text{Cl}}
\usepackage{mathtools}
\DeclarePairedDelimiter\ceil{\lceil}{\rceil}
\DeclarePairedDelimiter\floor{\lfloor}{\rfloor}


\usepackage{indentfirst}
\setlist{
    listparindent = \parindent,
    parsep = 6pt,
}

\makeatletter
\newsavebox\myboxA
\newsavebox\myboxB
\newlength\mylenA

\newcommand*\xoverline[2][0.9]{%
    \sbox{\myboxA}{$\m@th#2$}%
    \setbox\myboxB\null% Phantom box
    \ht\myboxB=\ht\myboxA%
    \dp\myboxB=\dp\myboxA%
    \wd\myboxB=#1\wd\myboxA% Scale phantom
    \sbox\myboxB{$\m@th\overline{\copy\myboxB}$}%  Overlined phantom
    \setlength\mylenA{\the\wd\myboxA}%   calc width diff
    \addtolength\mylenA{-\the\wd\myboxB}%
    \ifdim\wd\myboxB<\wd\myboxA%
       \rlap{\hskip 0.5\mylenA\usebox\myboxB}{\usebox\myboxA}%
    \else
        \hskip -0.5\mylenA\rlap{\usebox\myboxA}{\hskip 0.5\mylenA\usebox\myboxB}%
    \fi}
\makeatother

\printanswers

\begin{document}

\abovedisplayskip=12pt
\belowdisplayskip=12pt
\abovedisplayshortskip=7pt
\belowdisplayshortskip=10pt
\allowdisplaybreaks

\setlength{\parindent}{18pt}

\title{FRE-GY 6233: Assignment 2}
\author{Raymond Luo}
\date{\today}
\maketitle

\noindent {\bf Problem 1:}
\begin{enumerate}[label = (\roman*)]
    \item Suppose that $(\Omega, \mathcal{F}, \pp)$ is a probability space where $\Omega = \{a, b, c, d, e, f\}$, $\mathcal{F}$ is $\sigma$-algebra,
      and $\pp$ is uniform (so $\pp(a) = \pp(b) = \cdots = \frac{1}{6}$).
    \item Let $X, Y, Z$ be r.v. given by
      \begin{multline}
        X(a) = 1, X(b) = X(c) = 3, X(d) = X(e) = 5, X(f) = 7 \\
        Y(a) = Y(b) = 2, Y(c) = Y(d) = 1, Y(e) = Y(f) = 7 \\
        Z(a) = Z(b) = Z(c) = Z(d) = 3, Z(e) = Z(f) = 2
      \end{multline}
\end{enumerate}
Solve the follow questions:
\begin{enumerate}
  \item Write down $\sigma(X), \sigma(X), \sigma(Z)$. Are there any relationships between them?
    \begin{solution}
      \begin{multline*}
        \sigma(X) = \{\emptyset, \Omega, \{a\}, \{b,c\}, \{d,e\}, \{f\}, \{a,b,c\}, \{a,d,e\}, \{a,f\}, \\
            \{b,c,d,e\}, \{b,c,f\}, \{d,e,f\},
            \{a,b,c,d,e\}, \{b,c,d,e,f\}, \{a,d,e,f\}, \{a,b,c,f\}\} \\
        \sigma(Y) = \{\emptyset, \Omega, \{a,b\}, \{c,d\}, \{e,f\}, \{a,b,c,d\}, \{a,b,e,f\}, \{c,d,e,f\}\}\\
        \sigma(Z) = \{ \emptyset, \Omega, \{a,b,c,d\}, \{e,f\} \}
      \end{multline*}

    We note that $\sigma(Z) \subset \sigma(Y)$.
    \end{solution}
  \item Define a r.. $\ee[X \mid \sigma(Y)]$
    \begin{solution}
      As $\ee[X|Y] = \Sigma_{i \in \{1,2,7\}} \ee[X|Y = i]\id_{i}(y)$, we consider the following: \\
      If $Y = 1$, $\pp(Y^{-1}(1) = c) = \frac{\pp(c)}{\pp(c) + \pp(d)} = \frac{1}{2}, \pp(Y^{-1}(1) = d) = \frac{1}{2}$\\
      So, $\ee[X|Y = 1] = (X(c) \cdot \frac{1}{2} + X(d) \cdot \frac{1}{2}) = 3\cdot \frac{1}{2} + 5\cdot \frac{1}{2}
      = 4$ \\
      If $Y = 2$, $\pp(Y^{-1}(2) = a) = \frac{\pp(a)}{\pp(a) + \pp(b)} = \frac{1}{2}, \pp(X^{-1}(2) = b) = \frac{1}{2}$\\
      So, $\ee[X|Y = 2] = (X(a) \cdot \frac{1}{2} + X(b) \cdot \frac{1}{2}) = 2$ \\
      If $Y = 7$, $\pp(Y^{-1}(7) = e) = \frac{\pp(e)}{\pp(e) + \pp(f)} = \frac{1}{2}, \pp(Y^{-1}(7) = f) = \frac{1}{2}$ \\
      So, $\ee[X|Y = 7] = (X(e) \cdot \frac{1}{2} + X(f) \cdot \frac{1}{2}) = 6$ \\.
    \end{solution}
  \item Check directly the averaging property
    \begin{equation*}
      \ee[\ee[X \mid \sigma(Y)]] = \ee[X]
    \end{equation*}

    \begin{solution}
      $\ee[\ee[X \mid \sigma(Y)]] = \Sigma_{i \in \{1,2,7\}} \ee[X|Y=i] \cdot \pp[Y=i]$. As $\pp[Y=i] = \frac{1}{3}$ for $i \in \{1,2,7\}$, we have that
      $\ee[\ee[X \mid \sigma(Y)]] = \frac{1}{3}\cdot (4 + 2 + 6) = 4$. \\
      We then note that: \\
      $\ee[X] = \Sigma_{i \in \{a,b,c,d,e,f\}} \pp[i] \cdot X(i) = \frac{1}{6}(1+3+3+5+5+7) = \frac{24}{6} = 4$. \\
      From this we have checked the averaging property.
    \end{solution}
  \item Show directly (by calculating) that
    \begin{equation*}
      \ee[\ee[X \mid \sigma(Y)] \mid \sigma(Z)] = \ee[X \mid \sigma(Z)]
    \end{equation*}

    \begin{solution}
      We first show the RHS: \\
      Let us denote events $\omega_{z1} = \{a,b,c,d\}$, $\omega_{z2} = \{e,f\}$ to partition $\Omega$ to form $\sigma(Z)$.
      We then denote events $\omega_{x1} = \{a\}, \omega_{x2} = \{b,c\}, \omega_{x3} = \{d,e\}, \omega_{x4} = \{f\}$ to partition $\Omega$ to form $\sigma(X)$. \\
      We then note that $\omega_{z1}\cap\omega_{x1} = \{a\}$, $\omega_{z1}\cap\omega_{x2} = \{b,c\}$, $\omega_{z1}\cap\omega_{x3} = \{d\}$. We then have \\
      $\ee[X \mid \omega_{z1}] = X(\omega \in \{a\})\frac{\pp(\{a\})}{\pp(\omega_{z1})} + X(\omega \in \{b,c\})\frac{\pp(\{b,c\})}{\pp(\omega_{z1})} +
      X(\omega \in \{d\})\frac{\pp(\{d\})}{\pp(\omega_{z1})} = 1\cdot \frac{1}{4} + 3\cdot \frac{2}{4} + 5\cdot \frac{1}{4} = 3$ \\
      We also note that $\omega_{z2} \cap \omega_{x3} = \{e\}, \omega_{z2} \cap \omega_{x4} = \{f\}$ so that we have \\
      $\ee[X \mid \omega_{z2}] = X(\omega \in \{e\})\frac{\pp(\{e\})}{\pp(\omega_{z2})} + X(\omega \in \{f\})\frac{\pp(\{f\})}{\pp(\omega_{z2})}
      = 5\cdot \frac{1}{2} + 7 \cdot \frac{1}{2} = 6$ \\
      We write this as:
        \begin{equation*}
          \ee[X \mid \sigma(Z)] =
            \begin{cases}
              3, & \text{for } \omega \in \omega_{z1} = \{a,b,c,d\} \\
              6, & \text{for } \omega \in \omega_{z2} = \{e,f\} \\
            \end{cases}
        \end{equation*}
      We then look at the LHS: \\
      In part (ii), we had subtly defined $X \mid \sigma(Y)$ over events $\omega_{y1} = \{a,b\}, \omega_{y2} = \{c,d\}, \omega_{y3} = \{e,f\}$ that partition $\Omega$ to form $\sigma(Y)$.
      From that we recieved the following random variable:
        \begin{equation*}
          \ee[X \mid \sigma(Y)] =
            \begin{cases}
              2, & \text{for } \omega \in \omega_{y1} = \{a,b\} \\
              4, & \text{for } \omega \in \omega_{y2} = \{c,d\} \\
              6, & \text{for } \omega \in \omega_{y3} = \{e,f\} \\
            \end{cases}
        \end{equation*}
      We also noted that $\sigma(Z) \subset \sigma(Y)$. It follows that as $\omega_{y1} \cup \omega_{y2} = \omega_{z1}$ and $\omega{y_3} = \omega_{z3}$,
        \begin{align*}
          &\ee[X \mid \sigma(Y)] \mid \omega_{z1}] \\
            &= \big(\ee[X \mid \omega_{y1}]\mid{\omega_{z1}}\big)\cdot \frac{\pp(\omega_{y1}\cap \omega_{z1})}{\omega_{z1}}
            + \big(\ee[X \mid \omega_{y2}]\mid{\omega_{z1}}\big)\cdot \frac{\pp(\omega_{y2}\cap \omega_{z1})}{\omega_{z1}} \\
            &= 2 \cdot \frac{1}{2} + 4\cdot \frac{1}{2} = 3\\
\\
          &\ee[X \mid \sigma(Y)] \mid \omega_{z2}] \\
            &= \big(\ee[X \mid \omega_{y3}]\mid{\omega_{z2}}\big)\cdot \frac{\pp(\omega_{y3}\cap \omega_{z2})}{\omega_{z2}}
            = \ee[X \mid \omega_{z3}] = 6
        \end{align*}
      We then have:
      \begin{equation*}
        \ee[X \mid \sigma(Y)] \mid \sigma(Z)]  =
          \begin{cases}
            3, & \text{for } \omega \in \omega_{z1} = \{a,b,c,d\} \\
            6, & \text{for } \omega \in \omega_{z2} = \{e,f\} \\
          \end{cases}
      \end{equation*}
      So it is evident that LHS = RHS.

    \end{solution}

  \item Check if $X$ and $Y$, or $Y$ and $Z$ are independent under given probability.

    \begin{solution}
      It is straightforward to see that $X$ and $Y$ are not independent; otherwise, $\ee[X\mid \sigma(Y)] = \ee[X]$. We have shown that
      $\ee[X] = 4$, which is clearly not what we have shown to be $\ee[X\mid \sigma(Y)]$ in the problems above. \\
      To show that $Y$ and $Z$ are not independent, we proceed by using the definition of independence. We check that for some $\omega_{1} \in \sigma(Y),
      \omega_2 \in \sigma(Z)$, that $\pp(\omega_1 \cap \omega_2) \neq \pp(\omega_1)\cdot \pp(\omega_2)$. It is most evident that as $\omega_{z2} = \omega_{y3}$
      we would have $\pp(\omega_{z2} \cap \omega_{y3}) = \pp(\omega_{z2}) = \pp(\omega_{z2})\pp(\omega_{y3}) = \pp(\omega_{z2})^2$. As $\pp(\omega_{z2}) = \frac{1}{3}$,
      this cannot be true. As the two sigma-algebra are not independent, random variables $Y$ and $Z$ are not independent.

    \end{solution}
\end{enumerate}

\noindent {\bf Problem 2:} Prove Markov and Tchebyshev inequalities.

\begin{solution}
    \noindent {Markov Inequality:}
    For $\lambda, p > 0$ and nonnegative random variable $X$, we have:
    \begin{equation*}
      \pp(\omega: |X(\omega)| \geq \lambda) \leq \frac{1}{\lambda^p} \ee[|X|^{p}]
    \end{equation*}

    \begin{proof}
      We first by showing the above inequality for $p = 1$. This follows directly from the definition of expectation: \\
      If we fix $\lambda > 0$ and define set $A = \{\omega : X(\omega) \geq \lambda \}$
      \begin{multline*}
        \ee[X] = \int_{\rr} X(\omega) dP(\omega) = \int_{A} X(\omega) dP(\omega) + \int_{\rr \setminus A} X(\omega) dP(\omega)
        \\
        \geq \int_{A} X(\omega) dP(\omega) \geq \int_{A} \lambda dp(\omega) \text{ [this follows from the condition that $X(\omega) \geq \lambda$ in set $A$ ]} \\
        = \lambda \int_{A} dp(\omega) = a \pp(\omega: |X(\omega)| \geq \lambda). \\
        \Rightarrow \\
        \pp(\omega: |X(\omega)| \geq \lambda) \leq \frac{1}{\lambda^p} \ee[|X|^{p}]
      \end{multline*}
      To extend it to all other $p > 0$, we consider function $\phi(x) = |x|^p$.
      If $\phi$ is positive and non-decreasing, we have that $\pp[X \geq \lambda]
      \leq \pp[\phi(X) \geq \phi(\lambda)] \leq \ee[\phi(X)]/\phi(\lambda).$ As $\phi$ is positive and non-decreasing, we are done.
    \end{proof}

    \noindent {Tchebychev Inequality:}
    If $X$ is a r.v. with mean $\mu$ and variance $\sigma^2$, then

    \begin{equation*}
      \pp(\omega: |X(\omega) - \mu| \geq \lambda) \leq \frac{\sigma^2}{\lambda^2}
    \end{equation*}

    \begin{proof}
      The above inequality follows from Markov's inequality on $(X-\mu)$ with $p = 2$. We observe that
      $\pp(\omega: |X(\omega) - \mu| \geq \lambda) \leq \frac{\ee[|X - \mu|^2]}{\lambda^2}
      = \frac{\ee[X^2 - 2\mu\cdot X + \mu^2]}{\lambda^2} \\
      = \frac{1}{\lambda^{2}}\cdot \big( \ee[X^2] - 2\mu^{2} + \mu^2 \big)
      = \frac{1}{\lambda^{2}} \cdot (\ee[X^2] - \ee[X]^2) = \frac{\sigma^2}{\lambda^2}$
    \end{proof}
  \end{solution}


\noindent {\bf Problem 3:} Let $X$ be a r.v. and $\lambda > 0$. Prove that the following bound holds:
  \begin{equation*}
    \pp(X \geq \lambda) \leq \frac{\ee[e^{tX}]}{e^{\lambda t}}, \forall t>0
  \end{equation*}
Use Markov inequality.

  \begin{solution}
    We note that the function $\phi(x) = e^{tx}$ for $t > 0$ is a positive and non-decreasing function. We first have by Markov's inequality that
    $\pp(X \geq \lambda) = \pp(\phi(x) \geq \phi(\lambda)) \\
    \leq \frac{1}{\phi(\lambda)} \ee[\phi(X)] \Rightarrow \pp(X \geq \lambda) \leq \frac{\ee[e^{tX}]}{e^{\lambda t}}, \forall t>0$.
  \end{solution}

\end{document}
