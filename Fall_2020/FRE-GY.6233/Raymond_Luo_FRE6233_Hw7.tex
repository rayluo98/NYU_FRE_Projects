\documentclass[12pt,twoside, letter]{exam}
\usepackage{enumitem, kantlipsum}
\usepackage[margin=1in,left=1in,right=1in,top=1in,bottom=1in]{geometry}
\usepackage{graphicx,epstopdf}
\usepackage{amssymb,amsmath,amsfonts, amsthm}
\usepackage{wasysym}
\newtheorem{theorem}{Theorem}
\newtheorem{corollary}{Corollary}[theorem]
\newtheorem{lemma}[theorem]{Lemma}
\usepackage{hyperref}
\usepackage{tikz}
\usepackage{xstring}
\usetikzlibrary{calc}
\usepackage{ducksay}
\newtheorem{prop}{Proposition}

\theoremstyle{definition}
\newtheorem{definition}{Definition}

\usepackage{bbm}
\usepackage{verbatim}
\usepackage{bbold}
\usepackage{phfparen}
\usepackage{sets}
\newcommand{\nn}{\mathbb{N}}
\newcommand{\rr}{\mathbb{R}}
\newcommand{\cc}{\mathbb{C}}
\newcommand{\cb}{\mathcal{B}}
\newcommand{\ctau}{\mathcal{T}}
\newcommand{\co}{\mathcal{O}}
\newcommand{\zz}{\mathbb{Z}}
\newcommand{\ee}{\mathbb{E}}
\newcommand{\qq}{\mathbb{Q}}
\newcommand{\interior}{\text{Int}}
\newcommand{\pp}{\mathbb{P}}
\newcommand{\id}{\mathbbm{1}}
\newcommand{\Co}{\text{Co}}
\newcommand{\Cl}{\text{Cl}}
\usepackage{mathtools}
\DeclarePairedDelimiter\ceil{\lceil}{\rceil}
\DeclarePairedDelimiter\floor{\lfloor}{\rfloor}


\usepackage{indentfirst}
\setlist{
    listparindent = \parindent,
    parsep = 6pt,
}

\makeatletter
\newsavebox\myboxA
\newsavebox\myboxB
\newlength\mylenA

\newcommand*\xoverline[2][0.9]{%
    \sbox{\myboxA}{$\m@th#2$}%
    \setbox\myboxB\null% Phantom box
    \ht\myboxB=\ht\myboxA%
    \dp\myboxB=\dp\myboxA%
    \wd\myboxB=#1\wd\myboxA% Scale phantom
    \sbox\myboxB{$\m@th\overline{\copy\myboxB}$}%  Overlined phantom
    \setlength\mylenA{\the\wd\myboxA}%   calc width diff
    \addtolength\mylenA{-\the\wd\myboxB}%
    \ifdim\wd\myboxB<\wd\myboxA%
       \rlap{\hskip 0.5\mylenA\usebox\myboxB}{\usebox\myboxA}%
    \else
        \hskip -0.5\mylenA\rlap{\usebox\myboxA}{\hskip 0.5\mylenA\usebox\myboxB}%
    \fi}
\makeatother

\usepackage{float}
\floatstyle{boxed}
\restylefloat{figure}

\printanswers

\begin{document}


\abovedisplayskip=12pt
\belowdisplayskip=12pt
\abovedisplayshortskip=7pt
\belowdisplayshortskip=10pt
\allowdisplaybreaks

\setlength{\parindent}{18pt}

\title{FRE-GY 6233: Assignment 7}
\author{Raymond Luo}
\date{\today}
\maketitle

\noindent {\bf Problem 1}
  \par{Consider the same problem, as on slides 7 and 8 and generalize the strategy to replicate the call option with strike $K = 5,
  S_0 = 4, r = 0.25$ per period, probabilities of tail or head $p = q = \frac{1}{2}$}

\begin{equation*}
  S_1(H) = 8, S_1(T) = 2, S_2(HH) = 16, S_2(HT) = S_2(TH) = 4, S_2(TT) = 1
\end{equation*}

  \par{The option now expiring at period 2. You need to adjust the delta position in stock at time 1 (depending on the outcome of the
  first toss). Ultimately, as in the example in the lecture, the position in the stock and money market should perfectly replicate
  the option payoff along any path.}
  \\

  \begin{solution}
    We have the following call option payoffs:
    \begin{equation*}
      C_2(HH) = 11, C_2(HT) = C_2(TH) = C_2(TT) = 0
    \end{equation*}
    As according to the Fundamental Theorem of Asset Pricing, the extended market with our call option is arbitrage-free if it satisfies
    that the discounted value of the call option is a martingale. We then have:
    \begin{align*}
      C_1(H) = \ee^{\mathbf{P}}[C_2/(1+r)] = \frac{1}{1+r}\big[pC_2(HH) + qC_2(HT)\big] = \frac{4}{5}\cdot(\frac{1}{2}\cdot 11 + \frac{1}{2}\cdot 0) = \$ 4.4 \\
      C_1(T) = \ee^{\mathbf{P}}[C_2/(1+r)] = \frac{1}{1+r}\big[pC_2(TT)) + qC_2(TT)\big] = \frac{4}{5}\cdot(\frac{1}{2}\cdot 0 + \frac{1}{2}\cdot 0) = \$ 0
    \end{align*}
    And time 0 pricing of the call option at:
    \begin{equation*}
      C_0 = \ee^{\mathbf{P}}[C_1/(1+r)] = \frac{1}{1+r}\big[pC_1(H) + qC_2(T)\big] = \frac{4}{5}\cdot(\frac{1}{2}\cdot 4.4 + \frac{1}{2}\cdot 0) = \$ 1.76
    \end{equation*}
    From this, we can create a replicating portfolio by buying $\delta(s,t)$ shares of stock where $s$ is the price of the stock at time $t$.
    We then have:
    \begin{align*}
      2s \cdot \delta(s,t) + B(t)\cdot(1+r) &= v(2s, t+1) \\
      s/2 \cdot \delta(s,t) + B(t)\cdot(1+r) &= v(s/2, t+1) \\
      &\Rightarrow \\
      \delta(s,t) &= \frac{v(2s,t+1) - v(s/2, t+1)}{2s - s/2}
    \end{align*}
    where $v(s,t)$ is the value of call option at time $t$ when the stock price is at $s$. We then have:
    \begin{align*}
      \delta(4, 0) = \frac{4.4 - 0}{8 - 2} = 0 .73\\
      \delta(8, 1) = \frac{11 - 0}{16 - 4} = 0.92\\
      \delta(2, 1) = \frac{0 - 0}{4 - 1} = 0
    \end{align*}

  \end{solution}

\noindent {\bf Problem 2}
\par{Work with the BS PDE (3), slide 19}

\begin{enumerate}
  \item Show details how to get from BS equation (3) to the simpler equation (8) by changing variables and the function as in (7) on
    the slide 19. Take a proper care of the second derivative $\frac{\partial^2}{\partial x^2}$ through original derivatives
    $\frac{\partial^2}{\partial S^2}$ and $\frac{\partial}{\partial S}$
      \begin{solution}
        We take the following change of variables:
        \begin{equation*}
          S = Ke^x, t = T - \frac{\tau}{\frac{1}{2}\sigma^2}, c(t,S) = Kv(x, \tau)
        \end{equation*}
        So that we observe:
        \begin{align*}
          c(t,S) = Kv(\ln(\frac{S}{K}), \frac{\sigma^2}{2}(T-t))
        \end{align*}
        We then have the following by differentiating:
        \begin{align*}
          \frac{\partial c}{\partial t} = \frac{\partial \tau}{\partial t} \cdot \frac{\partial v}{\partial \tau}\\
          = \frac{-\sigma^2}{2}K \frac{\partial v}{\partial \tau} \\
          \frac{\partial c}{\partial S} = \frac{\partial x}{\partial S} \cdot \frac{\partial v}{\partial x}\\
          = \frac{K}{S} \frac{\partial v}{\partial x} \\
          \frac{\partial^2 c}{\partial S^2} = \frac{\partial}{\partial S} \frac{K}{S} \frac{\partial v}{\partial x} \\
          = \frac{K}{S^2}\frac{\partial^2 v}{\partial x^2} - \frac{K}{S^2} \frac{\partial v}{\partial x} \\
        \end{align*}
        We can then substitute the equations back into (3) to get:
        \begin{align*}
          LHS &= rc(t,S) = rKv(\ln(\frac{S}{K}), \frac{\sigma^2}{2}(T-t)) \\
          RHS &= c_t(t,S) + rSc_s(t,S) + \frac{1}{2} \sigma^2 S^2c_{ss}(t,S) \\
          &= \frac{-\sigma^2}{2}K v_\tau + rS \frac{K}{S} v_x + \frac{1}{2} \sigma^2 S^2 \frac{K}{S^2}\big(v_{xx} - v_x \big) \\
          &= \frac{-\sigma^2}{2}K v_\tau + rK v_x + \frac{\sigma^2}{2} K \big(v_{xx} - v_x \big) \\
          \Rightarrow \\
          rv &= \frac{-\sigma^2}{2} v_\tau + (r - \frac{\sigma^2}{2}) v_x + \frac{\sigma^2}{2} v_xx\\
          \frac{\partial v}{\partial \tau} &= \frac{\partial^2 v}{\partial x^2}
          + \bigg(\frac{2r}{\sigma^2} - 1\bigg)\frac{\partial v}{\partial x} - \frac{2r}{\sigma^2}v
        \end{align*}
        With $k_1 = \frac{2r}{\sigma^2}$, we get equation (8):
        \begin{equation*}
        \frac{\partial v}{\partial \tau} = \frac{\partial^2 v}{\partial x^2}
        + \bigg(k_1 - 1\bigg)\frac{\partial v}{\partial x} - k_1v
        \end{equation*}
      \end{solution}
  \item Show details how to get equation (9) from equation (8), and then the PDE (10) with initial condition (11)
    \begin{solution}
      We use a change of variable $v(\tau, x) = exp(\alpha x + \beta \tau)u(\tau, x)$ where $\alpha, \beta$ are constants.
      We then have:
      \begin{align*}
        LHS &= \frac{\partial v}{\partial \tau} \\
        &= \frac{\partial}{\partial \tau}exp(\alpha x + \beta \tau)u(\tau, x)
        + exp(\alpha x + \beta \tau)\frac{\partial u}{\partial \tau} \\
        &= exp(\alpha x + \beta \tau)\bigg(\beta u + \frac{\partial u}{\partial \tau} \bigg) \\
        RHS &= \frac{\partial^2 v}{\partial x^2} + \bigg(k_1 - 1\bigg)\frac{\partial v}{\partial x} - k_1v \\
        &= exp(\alpha x + \beta \tau)\bigg[(\alpha^2u + \frac{\partial^2 v}{\partial x^2}) + (k_1 - 1)
        \big(\alpha u + \frac{\partial u}{\partial x} \big) - k_1 u
        \big)\bigg] \\
        &\Rightarrow \\
        \beta u + \frac{\partial u}{\partial \tau} &= \alpha^2u + \frac{\partial^2 v}{\partial x^2} + (k_1 - 1)
        \big(\alpha u + \frac{\partial u}{\partial x} \big) - k_1 u
      \end{align*}
        Note that we want the equation to look a bit cleaner so we want terms to vanish by choosing:
        \begin{align}
          \begin{cases}
            u(\beta - \alpha^2 - (k_1 - 1)\alpha + k_1) = 0 \\
            \frac{\partial u}{\partial x} (2\alpha + (k_1 - 1)) = 0
          \end{cases}
        \end{align}
        so from equation (2) we choose $\alpha = \frac{-1}{2}(k_1 - 1)$, which means that from (1), $\beta = \frac{-1}{4}(k_1 + 1)^2$.
        Plugging this into our change of variable equation, we have that:\\
        $v(\tau, x) = exp(\frac{-1}{2}(k_1 - 1) x + \frac{-1}{4}(k_1 + 1)^2 \tau)u(\tau, x)$ so that we have differential equation: \\
        $\frac{\partial u}{\partial \tau} = \frac{\partial^2 v}{\partial x^2}$ with boundary condition \\
        $u(x,0) = e^{-\alpha x}\max\bigg(e^{x} - 1, 0 \bigg) = \max\bigg(e^{(1 - \alpha) x} - e^{-\alpha x}, 0 \bigg) \\
        = \max\bigg(e^{\frac{1}{2}(k_1 + 1)x} - e^{\frac{1}{2}(k_1 - 1) x}, 0 \bigg)$
    \end{solution}
\end{enumerate}

\noindent {\bf Problem 3}
\par{Consider the BS PDE (3) on slide 17 in the last lecture}

\begin{enumerate}
  \item Check that the following are the solutions of the equation.
    \begin{enumerate}
      \item $AS$
        \begin{solution}
          If we let $c(s,t) = AS$ we have that
          \begin{align*}
            \frac{\partial}{\partial t} c(s,t) = 0 \\
            \frac{\partial}{\partial s} c(s,t) = A \\
            \frac{\partial^2}{\partial s^2} c(s,t) = 0 \\
          \end{align*}
          So that we have:
          \begin{align*}
            RHS = c_t(t,S) + rSc_s(t,S) + \frac{1}{2} \sigma^2 S^2c_{ss}(t,S) \\
            = 0 + rS\cdot A + \frac{1}{2} \sigma^2 S^2 \cdot 0 \\
            = rS \cdot A = r c(s,t) = LHS
          \end{align*}
        \end{solution}
      \item $Ae^{rt}$, $A$ is a constant
        \begin{solution}
          If we let $c(s,t) = Ae^{rt}$ we have that
          \begin{align*}
            &\frac{\partial}{\partial t} c(s,t) = rAe^{rt} \\
            &\frac{\partial}{\partial s} c(s,t) = 0 \\
            &\frac{\partial^2}{\partial s^2} c(s,t) = 0 \\
          \end{align*}
          So that we have:
          \begin{align*}
            RHS &= c_t(t,S) + rSc_s(t,S) + \frac{1}{2} \sigma^2 S^2c_{ss}(t,S) \\
            &= rAe^{rt} + rS\cdot 0 + \frac{1}{2} \sigma^2 S^2 \cdot 0 \\
            &= rAe^{rt} = r c(s,t) = LHS
          \end{align*}
        \end{solution}
    \end{enumerate}
  \item Find the most general solution of the BS equation that has a special form
    \begin{enumerate}
      \item $V(S)$, so independent of time
    \end{enumerate}
    \begin{solution}
      Suppose $c(s,t) = V(s)$ for some function $V$. Then, we note that $\frac{\partial c}{\partial t}=0$ so that we have:
      \begin{align*}
        rV(S) = rSV_s(S) + \frac{1}{2} \sigma^2 S^2V_{ss}(S)
      \end{align*}
      We note that this is an ODE equation by:
      \begin{align*}
        \frac{\sigma^2}{2r} x^2f''(x) + x f'(x) - f(x) = 0 \\
        x^2f''(x) + k_1 xf'(x) - k_1 f(x) = 0
      \end{align*}
      We plug in general solution $f(x) = Ae^{iB\ln(x)}$ for constants $A, B$ so  that:
      \begin{align*}
        \bigg(x^2(\frac{-B^2}{x^2} - \frac{iB}{x^2}) + k_1x(\frac{iB}{x}) - k_1\bigg) \cdot Ae^{iB\ln(x)} = 0 \\
        -B^2 - iB + ik_1B - k_1 = 0\\
        (iB + k_1)(iB - 1) = 0\\
        B = -i \text{ or } B = k_1 i
      \end{align*}
      It follows that the general solution is: $f(x) = Ae^{\ln(x)} + Ae^{-k_1 \ln(x)}$.
    \end{solution}
\end{enumerate}

\noindent {\bf Problem 4}
  \par{Prove that the fundamental solution (13) is indeed a solution of the heat equation (12) for $\tau > 0$}
    \begin{solution}
      We want to show that $u_{\delta}(x, \tau) = \frac{1}{\sqrt{\pi\tau}}e^{-\frac{x^2}{4\tau}}$ is a solution to the heat equation.
      We have that
      \begin{align*}
        \frac{\partial u}{\partial \tau} = e^{-\frac{x^2}{4\tau}}\frac{\partial}{\partial \tau}\frac{1}{\sqrt{\pi\tau}} + \frac{1}{\sqrt{\pi\tau}}\frac{\partial}{\partial \tau}e^{-\frac{x^2}{4\tau}} \\
        = e^{-\frac{x^2}{4\tau}}\cdot (\frac{-1}{2\sqrt{\pi\tau^3}}) + \frac{1}{\sqrt{\pi\tau}}\cdot (\frac{x^2}{4\tau^2}) e^{-\frac{x^2}{4\tau}}
      \end{align*}
      And:
      \begin{align*}
        \frac{\partial^2 u}{\partial x^2} = \frac{1}{\sqrt{\pi\tau}} \frac{\partial}{\partial x} \bigg(\frac{-2x}{4\tau}e^{-\frac{x^2}{4\tau}}\bigg) \\
        = \frac{1}{\sqrt{\pi\tau}} \bigg(\frac{-2}{4\tau}e^{-\frac{x^2}{4\tau}} + \frac{4x^2}{16\tau^2}e^{-\frac{x^2}{4\tau}} \bigg)\\
        = e^{-\frac{x^2}{4\tau}}\cdot (\frac{-1}{2\sqrt{\pi\tau^3}}) + \frac{1}{\sqrt{\pi\tau}}\cdot (\frac{x^2}{4\tau^2}) e^{-\frac{x^2}{4\tau}}
      \end{align*}
      As $\frac{\partial u}{\partial \tau} = \frac{\partial^2 u}{\partial x^2}$, we have shown that (13) is a solution to the heat equation.
    \end{solution}


\end{document}
