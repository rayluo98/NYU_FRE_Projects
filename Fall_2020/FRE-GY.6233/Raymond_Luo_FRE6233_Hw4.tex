\documentclass[12pt,twoside, letter]{exam}
\usepackage{enumitem, kantlipsum}
\usepackage[margin=1in,left=1in,right=1in,top=1in,bottom=1in]{geometry}
\usepackage{graphicx,epstopdf}
\usepackage{amssymb,amsmath,amsfonts, amsthm}
\usepackage{wasysym}
\newtheorem{theorem}{Theorem}
\newtheorem{corollary}{Corollary}[theorem]
\newtheorem{lemma}[theorem]{Lemma}
\usepackage{hyperref}
\usepackage{tikz}
\usepackage{xstring}
\usetikzlibrary{calc}
\usepackage{ducksay}
\newtheorem{prop}{Proposition}

\theoremstyle{definition}
\newtheorem{definition}{Definition}

\usepackage{bbm}
\usepackage{verbatim}
\usepackage{bbold}
\usepackage{phfparen}
\usepackage{sets}
\newcommand{\nn}{\mathbb{N}}
\newcommand{\rr}{\mathbb{R}}
\newcommand{\cc}{\mathbb{C}}
\newcommand{\cb}{\mathcal{B}}
\newcommand{\ctau}{\mathcal{T}}
\newcommand{\co}{\mathcal{O}}
\newcommand{\zz}{\mathbb{Z}}
\newcommand{\ee}{\mathbb{E}}
\newcommand{\qq}{\mathbb{Q}}
\newcommand{\interior}{\text{Int}}
\newcommand{\pp}{\mathbb{P}}
\newcommand{\id}{\mathbbm{1}}
\newcommand{\Co}{\text{Co}}
\newcommand{\Cl}{\text{Cl}}
\usepackage{mathtools}
\DeclarePairedDelimiter\ceil{\lceil}{\rceil}
\DeclarePairedDelimiter\floor{\lfloor}{\rfloor}


\usepackage{indentfirst}
\setlist{
    listparindent = \parindent,
    parsep = 6pt,
}

\makeatletter
\newsavebox\myboxA
\newsavebox\myboxB
\newlength\mylenA

\newcommand*\xoverline[2][0.9]{%
    \sbox{\myboxA}{$\m@th#2$}%
    \setbox\myboxB\null% Phantom box
    \ht\myboxB=\ht\myboxA%
    \dp\myboxB=\dp\myboxA%
    \wd\myboxB=#1\wd\myboxA% Scale phantom
    \sbox\myboxB{$\m@th\overline{\copy\myboxB}$}%  Overlined phantom
    \setlength\mylenA{\the\wd\myboxA}%   calc width diff
    \addtolength\mylenA{-\the\wd\myboxB}%
    \ifdim\wd\myboxB<\wd\myboxA%
       \rlap{\hskip 0.5\mylenA\usebox\myboxB}{\usebox\myboxA}%
    \else
        \hskip -0.5\mylenA\rlap{\usebox\myboxA}{\hskip 0.5\mylenA\usebox\myboxB}%
    \fi}
\makeatother

\usepackage{float}
\floatstyle{boxed}
\restylefloat{figure}

\printanswers

\begin{document}


\abovedisplayskip=12pt
\belowdisplayskip=12pt
\abovedisplayshortskip=7pt
\belowdisplayshortskip=10pt
\allowdisplaybreaks

\setlength{\parindent}{18pt}

\title{FRE-GY 6233: Assignment 4}
\author{Raymond Luo}
\date{\today}
\maketitle

\noindent {\bf Problem 1: Generalized Breakout Question 1}\\
Let $X_t$ and $Y_t$ are stochastic processes, and martingales w.r.t filtration $\mathcal{F}_t$.
Show that a process $Z_t$ defined by
  \begin{equation*}
    Z_t = c_1X_t + c_2Y_t + c_3
  \end{equation*}
where $c_1, c_2, c_3$ are some constants, is a martingale w.r.t. $\mathcal{F}$

\begin{solution}
  We check the following:
  \begin{enumerate}[label =(\roman*)]
    \item $Z_t$ is integrable for each $t \in \mathcal{T}$: This follows immediately as $X_t$, $Y_t$ are martingales w.r.t to $\mathcal{F}_t$.
      Fix $t \in \mathcal{T}$, $X_t, Y_t$ are then integrable such that $\int |X_t| dt = a_1 < \infty, \int |Y_t| dt = a_2 < \infty$.
      Then $\int |Z_t| = \int |c_1X_t + c_2Y_t + c_3|dt \leq \int |c_1|\cdot|X_t| + |c_2|\cdot|Y_t| + |c_3| dt =  |c_1|\cdot a_1 + |c_2|
      \cdot a_2 + |c_3| < \infty$.
    \item $Z_t$ is adapted to the filtration $\mathcal{F}_t$: This follows immediately as $X_t$, $Y_t$ are martingales w.r.t to $\mathcal{F}_t$.
      We know that $X_t, Y_t$ are $\mathcal{F}_t$-measurable and linear combinations is a measurable function, $Z_t$ is $\mathcal{F}$-measurable as a composition
      of measurable functions.
    \item $Z_s = \ee[Z_t \mid \mathcal{F}_{s}] \forall s<t$: We have that $\ee[Z_t \mid \mathcal{F}_{s}] = \ee[c_1X_t + c_2Y_t + c_3 \mid \mathcal{F}_{s}]
      = c_1\ee[X_t \mid \mathcal{F}_{s}] + c_2\ee[Y_t \mid \mathcal{F}_{s}] + c_3
      = c_1X_s + c_2Y_s + c_3 = Z_s$
  \end{enumerate}
\end{solution}

\noindent {\bf Problem 2: Breakout Question 2}\\
We consider a "motivating" example and make a different choice of $y_i$:
  \begin{equation*}
    y_{i} = (t_{i} + t_{i-1})/2
  \end{equation*}
Show details and the answer for the integral calculations in this case.

\begin{solution}
  We evaluate the integral in our "motivating" example by using the Riemann-Stieltjes integral of
  the Wiener process with $g = W$. For the central difference choice of evaluating this integral we have that:
  \begin{align*}
    S_{n} &= \sum^n_{i=1} W(y_{i})[W(t_{i})-W(t_{i-1})] \\
    &= \frac{1}{2}\cdot 2\sum^n_{i=1} W(y_{i})[W(t_{i})-W(t_{i-1})] \\
    &= \frac{1}{2} \big[2\sum^n_{i=1} W(y_{i})[W(t_{i})-W(t_{i-1})]
    + 0 \big] *\\
    &= \frac{1}{2} \big[2\sum^n_{i=1} W(y_{i})[W(t_{i})-W(t_{i-1})]
    + \sum^{n}_{j=1} (W(y_{j}) - W(t_{j}))^2 - \sum^{n}_{j=1} (W(y_{j}) - W(t_{j-1}))^2 \big] \\
    &= \frac{1}{2} \big[2\sum^n_{i=1} W(y_{i})[W(t_{i}) - W(y_{i}) + W(y_{i}) -W(t_{i-1})]
    + \sum^{n}_{j=1} (W(y_{j}) - W(t_{j}))^2 - \sum^{n}_{j=1} W(y_{j})^2 + \\
    &+ \sum^{n}_{j=1} W(y_{j})^2 - \sum^{n}_{j=1} (W(y_{j}) - W(t_{j-1}))^2 \big] \\
    &= \frac{1}{2} \big[\sum^{n}_{j=1} (W(y_{j}) - W(t_{j}))^2 - 2\sum^n_{i=1} W(y_{i})[W({t_2}) - W(y_{i})]
    + \sum^{n}_{j=1} W(y_{j})^2  - \sum^{n}_{j=1} W(y_{j})^2 + \\
    & + 2\sum^n_{i=1} W(y_{i})[W(y_{i}) -W(t_{i-1})] - \sum^{n}_{j=1} (W(y_{j}) - W(t_{j-1}))^2 \big] \\
    &= \frac{1}{2} \big[\sum^{n}_{j=1} \big((W(t_j) - W(y_{j}))+W(y_{j})\big)^2
    - \sum^{n}_{j=1} \big((W(t_{j-1}) - W(y_{j}))+W(y_{j})\big)^2 \big] \\
    &= \frac{1}{2} \sum^{n}_{j=1} (W(t_{j})^2 - W(t_{j-1})^2) \\
    &= W(t)^2
  \end{align*}
\end{solution}

\noindent {\bf Problem 3: Isometry Property} \\
Show details of the proof that the cross terms in the double sum are zeros
  \begin{equation*}
    \ee(Y_{i}Y_{j}) = 0, i \neq j
  \end{equation*}
where $Y_{i} = Z_{i}\Delta_{i}W$

\begin{solution}
  We first fix $i, j \in \rr$ with $i < j$ so that:
  \begin{align*}
    &\ee[Y_{i}Y_{j}] = \ee[Z_{i}\Delta_{i}W \cdot Z_{j}\Delta_{j}W] = \ee[Z_{i}(W_{i}-W_{i-1}) \cdot Z_{j}(W_{j}-W_{j-1})] \\
    =& \ee[\ee[Z_{i}Z_{j}\cdot(W_{i}-W_{i-1})\cdot(W_{j}-W_{j-1}) \mid \mathcal{F}_{j-1}]] \text{ [by the tower property]} \\
    =& \ee[Z_{i}Z_{j}\cdot(W_{i}-W_{i-1})\big(\ee[W_{j}-W_{j-1} \mid \mathcal{F}_{j-1}]\big)]
    \text{ [as $Z_i, Z_j, (W_i-W_{i-1}), W_{j-1}$ are adaptive to $\mathcal{F}_{j-1}$ ]} \\
    =& \ee[Z_{i}Z_{j}\cdot(W_{i}-W_{i-1})\big(\ee[W_{j}\mid \mathcal{F}_{j-1}] -W_{j-1} \big)] \\
    =& \ee[Z_{i}Z_{j}\cdot(W_{i}-W_{i-1})\cdot 0] = 0
  \end{align*}
\end{solution}

\noindent {\bf Problem 4: Simple Processes}\\
Describe a simple process obtained from a linear combination of two simple processes:
For constants $c_1$ and $c_2$ and simple processes $C^{(1)}$ and $C^{(2)}$ on $[0,T]$, with (possibly)
different partitions $\tau_{n}^{(1)}$ and $\tau_{n}^{(2)}$
  \begin{equation*}
    C = c_1C^{(1)} + c_2C^{(2)}
  \end{equation*}
  \par{Thus, you need to define a partition for $C$ and its values. You can give an example as the illustration}

\begin{solution}
Suppose we have simple processes $C^{(1)}$ and $C^{(2)}$ on $[0,T]$, with partitions $\tau_{n}^{(1)}$ and $\tau_{n}^{(2)}$.
To describe simple process $C = c_1C^{(1)}+c_2C^{(2)}$, we need a partition $\tau = \tau_{n}^{(1)}\vee\tau_{n}^{(2)}$.
Here $\tau_{n}^{(1)}\vee\tau_{n}^{(2)}$ denotes the common refinement of $\tau_{n}^{(1)}$ and $\tau_{n}^{(2)}$.
More explicitly, given partitions $\tau_{n}^{(1)},\tau_{n}^{(2)}$ respectively defined by sequences
$a_0 = 0 < a_1 < a_2 < \cdots < a_{n} = T$ and $b_0 = 0 < b_1 < b_2 < \cdots < b_n = T$, our new partition
contains all the points in the two sequences re-numbered in order. So with a simple example, if $\forall j \in [1,n-1]$,
$b_{j-1} < a_{j} < b_{j}$, our resulting partition would be defined by a sequence
$a_0 = b_0 < a_1 < b_1 < a_2 < \cdots < b_{n-1} < a_n = b_n$. \\
We then note that the definition of our new simple process follows directly as the linear combination of the values of
the two contributing simple processes as defined on their respective partitions (as our common refinement is finer than
both original partitions in the sense that all points of both sequences are a part of our new sequence). Following the previous
example, if $C^{(1)},C^{(2)}$ are respectively defined by a sequence of random variables $X_{i},Y_{i}$ over their respective partitions
then:

\begin{align*}
  C(t) =
  \begin{cases}
    c_1X_n + c_2Y_n, &\text{ if $t=T$} \\
    c_2X_i + c_2Y_j, &\text{ if $b_{j-1} \leq t \leq b_{j}$ and $a_{i-1} \leq t \leq a_{i}$ for $i,j \in [1,n]$}
  \end{cases}
\end{align*}

\end{solution}

\noindent {\bf Problem 5: Wiener Integral}
Find a distribution of $Y$ defined by
  \begin{equation*}
    Y = \int^{T}_{1} sdW(s), T > 1
  \end{equation*}

\begin{solution}
  We first note that for a partition $\tau_{n}$ for some fixed $n \in \nn$:
  \begin{align*}
    Y &= \int^{T}_{1} sdW(s) = \sum^n_{j=1} t_{j}(W(t_j) - W(t_{j-1})))
  \end{align*}
  We then observe that as $W(t_j)-W(t_{j-1})$ are independently and normally distributed with finite variance, $Y$ is the countable sum of
  independent normal distributions each scaled by a corresponding factor of $t_{j}$.
  We conclude that this distribution should be normal as the countable sum of normal distributions
  so we observe the expectation and variance of the distribution. \\
  \begin{align*}
    \ee[Y] &= \ee[TW(T) + \frac{T-1}{n} \sum^{n}_{j=1} W(t_{j-1})] \\
    &= T\ee[W(T)] + \frac{T-1}{n} \sum^{n}_{j=1} \ee[W(t_{j-1})] = 0 \\
    Var[Y] &= \ee[Y^2] - \ee[Y]^2 \\
    &= \ee[Y^2] - 0 = \ee \bigg[\big(\int^{T}_{1} sdW(s)\big)^2\bigg] \\
    &= \ee \bigg[\big(\int^{T}_{0} sdW(s) - \int^{1}_{0} sdW(s) \big)^2\bigg] \\
    &= \ee \bigg[\big(\int^{T}_{0} sdW(s) \big)^2 -2\int^{1}_{0}sdW(s)\int^{T}_{1}sdW(s) + \big(\int^{1}_{0} sdW(s) \big)^2\bigg] \\
    &= \ee \bigg[\big(\int^{T}_{0} sdW(s) \big)^2\bigg]
    - 2\ee [\int^{1}_{0}sdW(s)\int^{T}_{1}sdW(s)]
    + \ee \bigg[\big(\int^{1}_{0} sdW(s) \big)^2\bigg] \\
    &= \ee \bigg[\big(\int^{T}_{0} sdW(s) \big)^2\bigg]
    + \ee \bigg[\big(\int^{1}_{0} sdW(s) \big)^2\bigg] =
  \end{align*}
    (Here the center term evaluates to zero as the two integrals are evaluated as
    the sum of increments of the wiener process scaled by time; however, the increments of the wiener process are independent and have
    mean zero, so this term disappears) \\
  \begin{align*}
    &= \ee[\int^{T}_{0} s^2 ds] + \ee[\int^{1}_{0} s^2 ds]\text{ (by Ito Isometry)} \\
    &= \ee[\frac{T^{3}}{3}] + \ee[\frac{1}{3}] = \frac{T^{3}}{3} + \frac{1}{3}
  \end{align*}
\end{solution}

\end{document}
