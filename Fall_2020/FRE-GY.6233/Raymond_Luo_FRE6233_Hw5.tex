\documentclass[12pt,twoside, letter]{exam}
\usepackage{enumitem, kantlipsum}
\usepackage[margin=1in,left=1in,right=1in,top=1in,bottom=1in]{geometry}
\usepackage{graphicx,epstopdf}
\usepackage{amssymb,amsmath,amsfonts, amsthm}
\usepackage{wasysym}
\newtheorem{theorem}{Theorem}
\newtheorem{corollary}{Corollary}[theorem]
\newtheorem{lemma}[theorem]{Lemma}
\usepackage{hyperref}
\usepackage{tikz}
\usepackage{xstring}
\usetikzlibrary{calc}
\usepackage{ducksay}
\newtheorem{prop}{Proposition}

\theoremstyle{definition}
\newtheorem{definition}{Definition}

\usepackage{bbm}
\usepackage{verbatim}
\usepackage{bbold}
\usepackage{phfparen}
\usepackage{sets}
\newcommand{\nn}{\mathbb{N}}
\newcommand{\rr}{\mathbb{R}}
\newcommand{\cc}{\mathbb{C}}
\newcommand{\cb}{\mathcal{B}}
\newcommand{\ctau}{\mathcal{T}}
\newcommand{\co}{\mathcal{O}}
\newcommand{\zz}{\mathbb{Z}}
\newcommand{\ee}{\mathbb{E}}
\newcommand{\qq}{\mathbb{Q}}
\newcommand{\interior}{\text{Int}}
\newcommand{\pp}{\mathbb{P}}
\newcommand{\id}{\mathbbm{1}}
\newcommand{\Co}{\text{Co}}
\newcommand{\Cl}{\text{Cl}}
\usepackage{mathtools}
\DeclarePairedDelimiter\ceil{\lceil}{\rceil}
\DeclarePairedDelimiter\floor{\lfloor}{\rfloor}


\usepackage{indentfirst}
\setlist{
    listparindent = \parindent,
    parsep = 6pt,
}

\makeatletter
\newsavebox\myboxA
\newsavebox\myboxB
\newlength\mylenA

\newcommand*\xoverline[2][0.9]{%
    \sbox{\myboxA}{$\m@th#2$}%
    \setbox\myboxB\null% Phantom box
    \ht\myboxB=\ht\myboxA%
    \dp\myboxB=\dp\myboxA%
    \wd\myboxB=#1\wd\myboxA% Scale phantom
    \sbox\myboxB{$\m@th\overline{\copy\myboxB}$}%  Overlined phantom
    \setlength\mylenA{\the\wd\myboxA}%   calc width diff
    \addtolength\mylenA{-\the\wd\myboxB}%
    \ifdim\wd\myboxB<\wd\myboxA%
       \rlap{\hskip 0.5\mylenA\usebox\myboxB}{\usebox\myboxA}%
    \else
        \hskip -0.5\mylenA\rlap{\usebox\myboxA}{\hskip 0.5\mylenA\usebox\myboxB}%
    \fi}
\makeatother

\usepackage{float}
\floatstyle{boxed}
\restylefloat{figure}

\printanswers

\begin{document}


\abovedisplayskip=12pt
\belowdisplayskip=12pt
\abovedisplayshortskip=7pt
\belowdisplayshortskip=10pt
\allowdisplaybreaks

\setlength{\parindent}{18pt}

\title{FRE-GY 6233: Assignment 5}
\author{Raymond Luo}
\date{\today}
\maketitle

\noindent {\bf Problem 1}
\par{Using the differentiation rules, find the following differentials:}

\begin{enumerate}[label = (\alph*)]
  \item $d(3W^{3}(t)  + e^{6W(t)})$
  \item $d(\frac{1}{t} \int^{t}_{0} W(u)du)$
  \item $d(t^3 cos(2W(t)))$
  \item $d(\frac{1}{t}\int^{t}_0 e^{W(u)}du)$
  \item $d(W(t)e^{W(t)})$
\end{enumerate}

\noindent {\bf Problem 2}
Let $Z(T) = \int^{T}_{0} W(u) du$ be the integrated BM, and $I(T) = \int^{T}_{0} tdW(t)$
  \begin{enumerate}[label = (\alph*)]
    \item Using integration by parts, find relationship bewteen $I(T)$ and $Z(T)$
    \item Using the relationship, compute the covariance between $I(T)$ and $Z(T)$
    \item Calculate the correlation between $I(T)$ and $Z(T)$
  \end{enumerate}

\noindent {\bf Problem 3}
Let $f(t) = e^{\alpha t}$, $g(x) = cos(x)$. Using integration by parts, calculate the stochastic integral
  \begin{equation*}
    \int^T_0 e^{\alpha t} cos(W(t)) dW(t)
  \end{equation*}
Which choice of $\alpha$ makes it very simple?

\noindent {\bf Problem 4}
Using the product of differentiation to calculate $d(X(t), Y(t))$, write down a general formula for integration by parts |
integrate both sides of a differential of a product of two stochastic processes $X(t)Y(t)$|

\begin{equation*}
  \int^T_0 Y(t) dX(t) = \cdots
\end{equation*}
Using the integration by parts, calculate 2 stochastic integrals:

\begin{enumerate}
  \item $\int^T_0 W(t)dW(t)$
  \item $\int^T_0 W^2(t)dW(t)$
\end{enumerate}


\end{document}
